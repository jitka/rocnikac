\chapter{Algorithms}

\section{Overview}
\TODO{proc prave PN}

\section{PN-search}

\TODO{pro jake typy her se da pouzit, jak budu kreslit obrazky}

\section{Enhancements of PN-search}

\subsection{Path from root}

Proof number search is best first search. It disadvantages is that searching for most-proving node takes
a lot of time. 

\newtheorem*{currentPath}{Definition}	
\begin{currentPath}
A~{\sl current path} is a~path from root to most-proving node which pn-search traversals.
\end{currentPath}

There are two enhancements which reduces the number of node traversals necessary to select
most-proving node. 

\subsubsection{Last changed node}

This enhancement was introduced in \cite{allis} (page 31-32).
In pn-search each iteration start at the root and traversals down. After developing
most-proving node we update proof and disproof numbers starting in most-proving and 
traversals into root. \TODO{obrazek: puvodni strom se zvirasnenou current path, a 
proof vlevo disproof vpravo s cervenym/kurzivou kde se zmenila hodnota}
So we traversals current path twice. 

There are two basic observation.

\begin{itemize}
\item When there is no change during node update, traversals can by stop even though we
aren't at root.
\item Current path of two consecutive iteration has the same beginning. They can by different 
only in part where proof or disproof number was updated.
\end{itemize}

\newtheorem*{currentNode}{Definition}	
\begin{currentNode}
A~{\sl current node} is a~node from current path. It is the highest changed node.
\end{currentNode}

\TODO{zkontrolovat jestli sedi nazvy fci}

Enhanced pn-search works similar like ordinary pn-search. It has one variable
called $currentNode$, at the beginning value of $currentNode$ is root. Each
$selectMostProvingNode()$ starts in $currentNode$. And each $updateAncestors()$
end at node where proof and disproof numbers aren't changed and set
$currentNode$ as the last changed node.

\subsubsection{DF-PN search}

\TODO{zarazky+hasovany}

\subsection{Transpositions in DAGs}

As we see in picture \TODO{3 obr,neco jen ... misto hran, nespojene, spojene
jen poradi, spojene i symetrie} of tic-tac-toe game tree.  We ale exploring one
game state many times. Basic enchantment is to join that nodes into one.

The problem is that we have game DAGs \footnote{There aren't cycles because in the
positional game all edges leads into position with more position occupied.}
instead of game tree and proof and disproof numbers
cannot by computed like above because one node can increase proof number of
another node more then on one \TODO{obr}. However $proofNumbers$ and $disproofNumbers$
computed by PN-search can by useful even though they can by higher then proof
and disproof number as they was defined \TODO{above}. 

We can use PN-search with few modification. When we generate children we look
if there is such node yet. We can use hash table for it. Second modification is
that we need update all parents. There could be problem if we use another
enchantments (see \TODO{chap}) but if we use ordinary PN-search it works. Allis
proof it in \cite{allis} (page 39-40).

One game state in turn $t$ can be exploring in the worst case $t!$ times
because position can by occupied in any order. So join them is good idea and
there exist many enchantments of this enchantment. They usually forgot on
symmetry, because in many games there are only few symmetry. For example in
chess there are only four symmetry and typically it is useless to count this
because white player has usually most chessman on his side. 

In this thesis we will try to use advantage of symmetry as much as it possible. However in clique
game we cannot joint all isomorph game state into one because find represent is hard problem.
So we will use almost representative node instead. See section \TODO{norm}.

\subsection{Heuristic}
\TODO{ty cisla na zacatku a zminit poradi synu}

\subsection{Problem  \TODO{obtečení}}

\subsection{Weak PN-search}
\TODO{pochlubit se ze to maji v clanku blbe}

\subsection{PN-set}

\subsection{1+\TODO{epsilon} trick}


\subsection{Deleting solved subtrees}
\TODO{+ nechani vysledku v cachy?}

\TODO{tady mozna PN2, DB}

