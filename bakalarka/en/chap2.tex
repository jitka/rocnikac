\chapter{Algorithms}

\section{Overview}
\TODO{proc prave PN}

\section{PN-search}

\TODO{pro jake typy her se da pouzit, jak budu kreslit obrazky}

\section{Enhancements of PN-search}

\subsection{Path from root}

Proof number search is best first search. It disadvantages is that searching for most-proving node takes
a lot of time. 

\newtheorem*{currentPath}{Definition}	
\begin{currentPath}
A~{\sl current path} is a~path from root to most-proving node which pn-search traversals.
\end{currentPath}

There are two enhancements which reduces the number of node traversals necessary to select
most-proving node. 

\subsubsection{Last changed node}

This enhancement was introduced in \cite{allis}\TODO{TODO} (page 31-32).
In pn-search each iteration start at the root and traversals down. After developing
most-proving node we update proof and disproof numbers starting in most-proving and 
traversals into root. \TODO{obrazek: puvodni strom se zvirasnenou current path, a 
proof vlevo disproof vpravo s cervenym/kurzivou kde se zmenila hodnota}
So we traversals current path twice. 

There are two basic observation.

\begin{itemize}
\item When there is no change during node update, traversals can by stop even though we
aren't at root.
\item Current path of two consecutive iteration has the same beginning. They can by different 
only in part where proof or disproof number was updated.
\end{itemize}

\newtheorem*{currentNode}{Definition}	
\begin{currentNode}
A~{\sl current node} is a~node from current path. It is the highest changed node.
\end{currentNode}

\TODO{zkontrolovat jestli sedi nazvy fci}

Enhanced pn-search works similar like ordinary pn-search. It has one variable
called $currentNode$, at the beginning value of $currentNode$ is root. Each
$selectMostProvingNode()$ starts in $currentNode$. And each $updateAncestors()$
end at node where proof and disproof numbers aren't changed and set
$currentNode$ as the last changed node.

\subsubsection{DF-PN search}


\subsection{Transpositions in DAGs}
\TODO{ a to jak ze se do jednoho vrcholu da dostat vice zpusoby tak simetrie}

\TODO{zakladni,DF+hasovany}

\subsection{Problem  \TODO{obtečení}}

\subsection{Heuristic}
\TODO{ty cisla na zacatku a zminit poradi synu}

\subsection{Weak PN-search}

\subsection{PN-set}

\subsection{1+\TODO{epsilon} trick}


\subsection{Deleting solved subtrees}
\TODO{+ nechani vysledku v cachy?}

\TODO{tady mozna PN2, DB}

