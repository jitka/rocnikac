\chapter{Experimental results}

In this chapter we describe directly our solver and results
illustrating which enhancements help with solving the clique
game. 

\section{Clique game} 

We solved the clique game for $K=4, N=6$ and $ N=7$. In this section we
describe the partial game trees which were created during solving that games. We
use \sec{weak PN-search} and \sec{transposition to DAG} with normalising function
\sec{sorting by vertex and neighbours degrees} and \sec{no free winning set}.
We call this combination \com{K4Weak}.
 
%N 7 M 22
%CACHE_SIZE 16777216, CACHE_PATIENCE 100, MAXNODES 6000000
%DEBUG WEAK STATS NORM2 TURNDDELETECHILDRENST 0 NOFREEK4 UPDATEANCESTORS2 

\begin{table} 
\centering
\begin{tabular}{l|c|c|c}
level & \sec{all} & \sec{created} & \sec{solved} \\
\hline
$\sum$ & 10 165 779& 1 241 (0,12)& 110\\
0& 1& 1 (0, 0)& 1\\
1& 16& 1 (0, 0)& 1\\
2& 240& 2 (0, 0)& 1\\
3& 1 680& 7 (0, 0)& 6\\
4& 10 920& 26 (0, 0)& 5\\
5& 43 680& 73 (0, 0)& 28\\
6& 160 160& 190 (0, 0)& 14\\
7& 400 400& 306 (0, 0)& 31\\
8& 900 900& 273 (0, 0)& 23\\
9& 1 441 440& 363 (0, 12)& 0\\
10& 2 018 016& 0& 0\\
11& 2 018 016& 0& 0\\
12& 1681 680& 0& 0\\
13& 960 960& 0& 0\\
14& 411 840& 0& 0\\
15& 102 960& 0& 0\\
16& 12 870& 0& 0\\
\end{tabular}
\caption{Properties of the game tree for (6,4)-clique game.}
\label{stats6}
\end{table}

\begin{table}
\centering
\begin{tabular}{l|c|c|c}
level & \sec{all} & \sec{created} & \sec{solved} \\
\hline $\sum$ & 6 377 181 825& 65 979 (92, 605)& 2 356\\
0& 1& 1 (0, 0)& 1\\
1& 22& 1 (0, 0)& 1\\
2& 462& 2 (0, 0)& 2\\
3& 4 620& 7 (0, 0)& 7\\
4& 43 890& 29 (0, 0)& 7\\
5& 263 340& 107 (0, 0)& 44\\
6& 1 492 260& 396 (0, 0)& 26\\
7& 5 969 040& 1 247 (0, 0)& 242\\
8& 22 383 900& 3 812 (0, 0)& 76\\
9& 62 674 920& 9 108 (0, 49)& 752\\
10& 162 954 792& 16 938 (0, 96)& 226\\
11& 325 909 584& 17 303 (9, 113)& 452\\
12& 597 500 904& 8 252 (0, 119)& 293\\
13& 853 572 720& 7 280 (43, 203)& 159\\
14& 1 097 450 640& 825 (0, 0)& 22\\
15& 1 097 450 640& 672 (40, 25)& 46\\
16& 960 269 310& 0& 0\\
17& 640 179 540& 0& 0\\
18& 355 655 300& 0& 0\\
19& 142 262 120& 0& 0\\
20& 42 678 636& 0& 0\\
21& 7 759 752& 0& 0\\
22& 705 432& 0& 0\\
\end{tabular}
\caption{Properties of the game tree for (7,4)-clique game.}
\label{stats7}
\end{table}

In the first case ($N=6$) the algorithm needs 337 iterations to solve the game. In the second
case the algorithm needs 16489 iterations and solves the game in three seconds on a desktop PC
(AMD Athlon 3GHz).

Tables~\ref{stats6} and~\ref{stats7} show the properties of the game tree for each level.

The column \sec{all} illustrates how many game states could exist on some level
$L$. This number was computed as the number of edge-colored complete graphs
with $N$ nodes with $\lfloor L / 2 \rfloor$ red edges and $\lfloor (L-1) / 2
\rfloor$ blue edges. Numbers from the first column are not exact because they
count also game states which have for example two $K_4$ with the same color
which cannot happen in a real game.

In the column \sec{created} we indicate the number of nodes which were actually created on each level. 
In the parentheses we indicate how many of the nodes were first evaluated as \val{false} respectively
marked as a~threat. There was no node which was first evaluated as \val{true} --
this is a~good sign, because nodes with value \val{true} never help to disprove anything and
the algorithm avoids them.

In the column \sec{solved} we indicate how many nodes were solved on level $L$.
All of them were solved as an~\val{false}.

\section{Comparing enhancements}

In this section we compare the changes caused by various enhancements, as measured
by several tests. We prepare some combination of enhancement and describe each
whole combination by one name.

\medskip

The criteria we use for comparison: 
\begin{itemize*}
\item time needed to solve the game (in seconds)
\item space which the program uses (number of created nodes)
\item number of iterations 
\item level of the deepest developed node 
\end{itemize*}

The number of created nodes is a representative value for space requirements when
we do not use any enhancement which deletes nodes.
In the time test we set the time limit for solving the game to 60 seconds.

\subsection{Weak PN-search} 

The enhancement \sec{weak PN-search}~\ref{weak} helps in every combination with
other enhancements which we tried. The results are in Table~\ref{statsWeak}.

\begin{table}
\centering
\begin{tabular}{c|l|c|c|c|c}
$N$ & name & time & iterations & created nodes & max. level \\
\hline
6 & \com{K4} & 1.41s & 413 & 1 406 & 10\\
6 & \com{K4Weak} & 1.03s & 337 & 1 241 & 8\\
\hline
7 & \com{K4} & 12.34s & 38 716 & 128 256 & 15\\
7 & \com{K4Weak} & 2.79s & 16 489 & 65 979 & 14\\
\hline
6 & \com{basic} & 5.38s & 5 541 & 6 353 & 14 \\
6 & \com{weak} & 2.48s & 4 237 & 6 079 & 14 \\
\end{tabular}
\caption{Comparing \sec{weak PN-search} variants.}
\label{statsWeak}
\end{table}


\subsubsection{\com{K4Weak} and \com{K4}} 

These two combinations use \sec{PN-search} with \sec{last changed nodes},
\sec{no free winning set} and normalise function \sec{sorting by vertex and
neighbours degrees}. Basic also uses \sec{weak PN-search}.

\subsubsection{\com{weak} and \com{basic}}

These two combinations are similar to previous ones but they both do not use the enhancement
\sec{no free winning set}. They both do not solve game for $N=7$ within the time limit.

\subsection{Deleting solved subtrees} \label{deletesolved2} 

This enhancement~\ref{deletesolved} does not help, not even when we use it only
with an ordinary \sec{PN-search}. It does not even considerably save
memory.\footnote{In this case we measured actual memory allocation, not number
of created nodes.}

The reason is that most of the time the game tree is being developed and not many
nodes are being solved. In the later part of the game when nodes are being
solved and deleted we do not need the memory which is saved. The result is that
when we run the program with and without this enhancement it needs approximately
the same amount of memory and program with \sec{delete solved subtrees} runs
longer. For this reason it is now not implemented in our code. Other
reason is that it causes many complications when we use it together with
\sec{transposition to DAG}, because \sec{deleting solved subtrees}
can delete node $n$ which we might need because there is another path to $n$. 

\subsection{No free winning set}

This enhancement radically decreases the level of the deepest developed node.
The number of nodes in subsequent levels of the complete game tree exponentially 
rises. For this reason the enhancement \sec{no free winning set} is
necessary to solve the  clique game for $N=7$. Results for $N=6$ are in
Table~\ref{statsNofreeK4}.

\begin{table}
\centering
\begin{tabular}{c|l|c|c|c|c}
$N$ & name & time & iterations & created nodes & max. level \\
\hline
6 & \com{weak} & 2.48s & 4 237 & 6 079 & 14 \\
6 & \com{K4Weak} & 1.03s & 337 & 1 241 & 8\\
\hline
6 & \com{basic} & 5.38s & 5 541 & 6 353 & 14 \\
6 & \com{K4} & 1.41s & 413 & 1 406 & 10\\
\end{tabular}
\caption{Comparing \sec{no free winning set} variants.}
\label{statsNofreeK4}
\end{table}

\subsection{Ordering of sons} 

This enhancement is used in each variant of our solver. If we want not to use
it we must randomly reorder sons. That is happening
when we sort sons by their hash in Section~\ref{delpar}.

A~first heuristic is when we use the ordinary \sec{PN-search}
and $developNode()$ creates sons this way that it try adding the new edge in
alphabetical order.

A~second heuristic is when we use \sec{PN-search} with \sec{transposition to DAG}
with normalising function \sec{sorting by vertex degree} or \sec{sorting by
vertex and neighbours degrees}. We first try colored the new
edge to area with low degree. It is good for the second player when all
vertices have similar degree. When the second player has draw strategy most node
are solved as \val{false} so the ordering of sons of \node{or} is not important 
node because we must solve them all.  Ordering of \node{and} node is good so
this heuristic works.

We try one other heuristic. When we are creating sons by adding the edge $e$ we
count the free winning sets including $e$ by function in Section~\ref{freeK4}.
Then we sort the sons so that the leftmost one has the largest number. This heuristic was
worse then second so currently it is not implemented.


\subsection{Deleting parallel edges} \label{delpar}
As we explain in Section~\ref{multi-edges} there could be parallel edges if we use the enhancement
\sec{transposition to DAG}. We implemented \sec{deleting parallel edges}, that
is we delete duplicate sons in the function \fce{updateAncestors()}. We
first create all the sons then normalise then, second we sort them by their hash and then check if nodes
with the same hash are actually same, then delete duplicities. 

However, after this the ordering of sons is different and it causes the partial
game tree to be different. This could be a disadvantage and could overweight
described advantages. The differences of the partial game trees are the reason
why we do not measure this enhancement.

\subsection{DF-PN-search}

\sec{DF-PN-search} was described in Section~\ref{dfpn}. In our case it does not help as
much as we expected. The reason is that it cannot be easily combined with
\sec{no free winning set}. 

To check if there are free $K_4$'s in a node $n$ is a slow operation. To make it faster we
postpone it to the time when $n$ is developed. Then when we create a son by
claiming an edge $e$ we check if there is a~free $K_4$ containing $e$ by the
function described in Section \ref{freeK4}. When there is no free $K_4$ for any
$e$, we assign the value \val{false} to $n$. When we use \sec{postpone testing}
together with \sec{DF-PN-search} the algorithm continues searching the subtree
of $n$ even if $n$ was solved.

\sec{no free winning set} helps more then \sec{DF-PN-search}. For this reason
\sec{DF-PN-search} is not included in the fastes variant. We show how it helps
when we use it for $N=6$ without \sec{no free winning set} in Table
\ref{statsDFPN}.

\begin{table}
\centering
\begin{tabular}{c|l|c|c|c}
$N$ & name & time & iterations & created nodes \\
\hline
6 & \com{weak} & 1.34s & 4 237 &  6 079 \\
6 & \com{dfpn} & 0.96s & 1 018 & 2 228 \\
\end{tabular}
\caption{Comparing \sec{DF-PN-search} variants.}
\label{statsDFPN}
\end{table}

%\subsection{Cache}
%We use \ODO{nazev-prosim poradte.. srustajici retizky} cache. When we are adding new node to
%cache we check in place with number equal to nodes hash. If this place is free
%we insert the node here. If not we check if some of next $c$ places is free and
%if so we insert the node on the first free place. If none of the places is free we
%must delete some of this $c$ nodes. When we are searching for some node we
%start checking on place with number equal to nodes hash and continue until we
%found it or we found free place or we search $c$ places. In last two cases the
%node is not in cache. 

\section{More about our implementation}

Our software was created in order to solve the clique game for $K=4$ and as high
$N$ as possible. We knew that there is no chance to solve it for all $N$ up
to 17 because just to check if a~strategy is winning needs checking so many
states that it could not be done. 
 
Our code is in C language and you can find it on
github\footnote{https://github.com/jitka/rocnikac/tree/master/src}
\TODO{licence - i udelat}.

During the creation of this project we need to test if some heuristic helps or
not.  First we tried to implement something and later we decided whether to
keep it or delete it. Later we decided to enclose enhancement into ``ifdefs''.
The result is that we can try combinations of more enhancements and we can
compare them easily.  There are thousands of possible combinations, but not all
of them works. List of combinations which were tested is in Makefile. 

\subsection{Implemented enhancements}

We implemented almost all the enhancements which are mentioned in this thesis. 
The enhancements which we do not compare helped in all the cases and currently it is
not possible to switch them off in our code.

We implemented all the enhancements from the second chapter and the first four
enhancements from the third chapter. We use the last described variant of
\fce{updateAncestors()} and \fce{selectMostProving()}. We did not implemented
enhancements \sec{Developing a~node without creating sons} from
Section~\ref{lastenh}. For storing nodes we use \sec{adjacency matrix in line}.
We implemented the first three normalise functions described in Section \ref{norm} --
the third normalize function \sec{Sorting by vertex and neighbours degrees} is the best
in terms of efficiency and still fast enough.

\subsection{The fastest variant}

Our fastest variant is \sec{K4Weak}. This  variant is \sec{weak PN-search}
with enhancements \sec{last changed nodes}, \sec{no free winning set} and normalise function
\sec{sorting by vertex and neighbours degrees}.

Most of the developed nodes are in levels in a middle of the complete game tree, see
Tables~\ref{stats6} and~\ref{stats7}. 

Solver spends most time by developing nodes. The program spends $21\%$ of
execution time in \fce{cacheFind()} and at least another $23\%$ in
\fce{normalise()} function which are both used in \fce{developNode()}.  Most of
the rest time is consumed by different functions for working with game state
which are mainly used also in \fce{developNode()}.

Only $6\%$ is consumed by \fce{updateAncestor} and less then $1\%$ by
\fce{selectMostProving()}. This is not surprise because in 13497 of 16489
iterations the new most proving node is a son of the previous one.

Time needed to solve the root depends on the size of the partial game tree.
When we want to make the solver faster we should try to reduce the size of the
partial game tree.

