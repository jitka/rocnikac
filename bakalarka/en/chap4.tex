\chapter{Software results}

In this chapter we will talk directly about our softwere. We give some
statistic which ilustrate which ehantments helps with solving clique
game. 

\section{ Clique game }
We solved clique game for $K=4, N=6$ and $ N=7$. In this section we
describe partial game trees which was created during solving that
games. We use \sec{weak pn-search} \ref{weak} and \sec{transposition
into DAG} \ref{DAG} with norm function \ref{norm2} and \TODO{\sec{
nofreeK4}} \ref{nofreeK4}.
 
%N 7 M 22
%CACHE_SIZE 16777216, CACHE_PATIENCE 100, MAXNODES 6000000
%DEBUG WEAK STATS NORM2 TURNDDELETECHILDRENST 0 NOFREEK4 UPDATEANCESTORS2 

\begin{table} 
\begin{tabular}{l|c|c|c}
level & \sec{all} & \sec{created} & \sec{solved} \\
\hline
$\sum$ & 10165779& 1241 (0,12)& 110\\
0& 1& 0 (0,0)& 1\\
1& 16& 1 (0,0)& 1\\
2& 240& 2 (0,0)& 1\\
3& 1680& 7 (0,0)& 6\\
4& 10920& 26 (0,0)& 5\\
5& 43680& 73 (0,0)& 28\\
6& 160160& 190 (0,0)& 14\\
7& 400400& 306 (0,0)& 31\\
8& 900900& 273 (0,0)& 23\\
9& 1441440& 363 (0,12)& 0\\
10& 2018016& 0 (0,0)& 0\\
11& 2018016& 0 (0,0)& 0\\
12& 1681680& 0 (0,0)& 0\\
13& 960960& 0 (0,0)& 0\\
14& 411840& 0 (0,0)& 0\\
15& 102960& 0 (0,0)& 0\\
16& 12870& 0 (0,0)& 0\\
\end{tabular}
\caption{Property of game tree of the clique game ($N=6,K=4$).}
\label{stats6}
\end{table}

\begin{table}
\begin{tabular}{l|c|c|c}
level & \sec{all} & \sec{created} & \sec{solved} \\
\hline $\sum$ & 6 377 181 825& 65 979 (92, 605)& 2 356\\
0& 1& 0 (0, 0)& 1\\
1& 22& 1 (0, 0)& 1\\
2& 462& 2 (0, 0)& 2\\
3& 4 620& 7 (0, 0)& 7\\
4& 43 890& 29 (0, 0)& 7\\
5& 263 340& 107 (0, 0)& 44\\
6& 1 492 260& 396 (0, 0)& 26\\
7& 5 969 040& 1 247 (0, 0)& 242\\
8& 22 383 900& 3 812 (0, 0)& 76\\
9& 62 674 920& 9 108 (0, 49)& 752\\
10& 162 954 792& 16 938 (0, 96)& 226\\
11& 325 909 584& 17 303 (9, 113)& 452\\
12& 597 500 904& 8 252 (0, 119)& 293\\
13& 853 572 720& 7 280 (43, 203)& 159\\
14& 1 097 450 640& 825 (0, 0)& 22\\
15& 1 097 450 640& 672 (40, 25)& 46\\
16& 960 269 310& 0& 0\\
17& 640 179 540& 0& 0\\
18& 355 655 300& 0& 0\\
19& 142 262 120& 0& 0\\
20& 42 678 636& 0& 0\\
21& 7 759 752& 0& 0\\
22& 705 432& 0& 0\\
\end{tabular}
\caption{Property of game tree of the clique game ($N=7,K=4$).}
\label{stats7}
\end{table}

In the first case ($N=6$) algorithm does 337 iteration to solve the game. It the second
case algorithm does 16489 iteration and solve it in tree second.

Tables \ref{stats6} and \ref{stats7} show information about levels in trees.

The column \sec{all} illustrate how many game state could exist on some level $L$. This
number was computed as a number of edge coloring graph with $N$ nodes with $L$
\fce{div} 2 red edges and $(L-1)$ \fce{div} 2 blue edges. So numbers from the first
column are not exactly right because they count game states which have for example
two $K_4$ with same color which cannot happen in real game.

In the column \sec{created}  are number of nodes which was really created on each level. 
In the bracket behind there how many of them was first evaluated as a \value{false} respectively
how many of them are thread. There was no node which was first evaluated as \value{true},
this is a god mark, because nodes with value \value{true} never helps to disprove anything and
algorithm se jim vyhne.

In the column \sec{solved} there is how many of nodes was solved on its level.
All of them was solved as a \value{false}.

\section{ \TODO{Meření zlepšení} }

\TODO{obecny kec o mereni
nezapomenout originalni cisla
}

\subsection{df}
\TODO{ze df-pn moc nepomohlo ale usetrilo pamet
df-proc mi moc nepomuze ze nezalezi na ceste ale na norm 
nefunguje nofreeK4
}

\subsection{Weak pn-search}
\TODO{
zkusit jine kombinace \^2
}

\subsection{Deleting solved subtrees} 
\TODO{nepomohlo, ted uz neni}

\subsection{\TODO{no free K4}}

\subsection{Ordering of sons} 
\TODO{nepomohlo, lepsi heuristiky?}

\subsection{Deleting multiple edges}

\subsection{Cache}

\subsection{\fce{selectMostProvingNode()} and \fce{updateAncestors()} }
\TODO{jak dlouho za jakych okolnosti trvna select update}

\subsection{\TODO{mereni obteceni}}

\section{ More about our implementation }

Our software was created in order to solve clique game for $K=4$ and as higher
$N$ as is possible. We knew that there is no chance to solve it for all $N$ up
to 17 because only check if winning strategy is right needs to check so many
states that it could be done. 
 
We chose language C. Our code is on github \TODO{jak se sazi adresa?} and
it is under \TODO{licence - i udelat}.

During crating this project we need to test if some heuristic helps or not.
First we try to implemented something and after we decide if let it there or if
deleted it. Later we decided to write enhancement into ifdefs. The result is
that we can try combination of more enhancement and we can compare them.
There are thousands of combination. Not all of them works perfectly. List
of combination which was tested is in documentation of the project. 

\subsection{ \TODO{nejrychlejsi verze} }

\TODO{kde je pomala}

\section{ Futher work }

\TODO{ i na teorii 
tabulky- zbytecne moc vytvorene
}
