\chapter{\TODO{Application on clique game}}

\TODO{kec 
to co jsem vymyslela
nejen clikova ale vsechny positional
}

\section{Other enhancements}
Here we will introduce some other enhancements of pn-search which we found and
implemented. We will also describe some solutions for problems which arise 
when we combine several enhancements together. 

\subsection{Ordering of sons} \label{ord}

Ordering of sons of some node could by important. When there are two or more
sons with minimal (dis)proof number we usually choose leftmost but there could
be better variant so we order sons by some heuristic.

\subsection{Deleting solved sons}

When we proof son of \node{or} node we proof also the node. When we disproof
son of \node{or} node and symmetrically when we proof son of \node{and} node it
is also useful and we can delete the relationship between them. The node will
have less sons so the sum or max or min of numbers could be counted quicker and
update ancestors will be also quicker. Bigger advantage happens when we use the
weak pn-search \ref{weak}\TODO{<-} too which depends on branching factor and it decreases.

\subsection{Deleting multiple edges }

When we use transposition into DAG multiple edges may be created, for example
on the beginning of solution clique game. Root has $N \choose 2$ sons. Sons are
$K_N$ with one edge colored. When we norm them they will all be the same and so
we have $N \choose 2$-multiple edge from root there. It is good idea to
change multi-edge to single edge.

\subsection{Variants of updateAncestors}

There are tree enhancements which influence updateAncestors function. First is
using last changed node \ref{last} to shorten the path. Second is df-pn-search
\ref{dfpn} even if there is no function named updateAncestors and it is
updating lazily. It updates the most proving node and the path up when coming
from recursion. Third is transposition into DAG \ref{DAG}. We found no article
considering changes in updateAncestor when node has more then one parent.

When we use ordinary updateAncestors even if we use transposition into DAG
\ref{DAG} the most proving node could be in another place than where algorithm
tries to find it. \obr{ 10) obrazek obteceni}. \footnote{This isn't as big
problem as it seems to be. In worst case we didn't solve the root. It couldn't
happen that the algorithm returns wrong answer. \TODO{urezat si nad sebou vetev
;-)}} To avoid it we must use wider updateAncestors and better
selectMostProvingNode as we will show in next subsection. 

Simple way how to define wider updateAncestors is recursively. Function updates
a node and if it changed at least one of proof and disproof numbers we call the
function recursively on nodes parents. It could happen that some nodes are
updated many times. So its parents are updated also many times and the number
of updates is growing with the number of possible paths. 

There are two better solutions. First, we can update each node on some level
and create list of parents which need to be updated. After we dispose with
duplicities we do the same with higher level. Second, we can store another
information --- in which iteration of pn-search was the node last updated. In
this case we can update recursively and stop when we are going to update some
node twice. 

\TODO{slo by to paralelne}

\subsection{Variants of selectMostProvingNode}

As you can see \obr{10} it could happen that some numbers are changed on
current path above unchanged node. So observation from section Last changed
node \ref{last} doesn't hold and it can happen that algorithm select most
proving node wrongly. There is a example \obr{15}. There are few solution which
we found.

\subsubsection{Start from root}

Simplest solution is to not use enhancements Last changed node \ref{last}
or df-pn-search \ref{dfpn}. In each iteration start search from root.

\subsubsection{Start at the highest changed level}

We can use this solution when we have current path in stack, which usually
happens when we use recursion. Function updateAncestors will compute level of
highest changed node. We delete so many nodes from the stack that stack has
length equal to the level of highest changed node. In next iteration we can
start searching from the node on \TODO{neni to naopak?} top of stack. This way
we will find same most proving node as when we are searching from root. We can
use same argument as in Last changed node \ref{last} to prove it.

\subsubsection{Remember current path}

In the best solution we found we hold whole current path in array. In every
node which is updated we test if it is in current path which can be done
quickly.  Level of updated node can by found as a number of positions
\footnote{ position in this case means position in the game state of positional
game} which are occupied or we can save this information in node. Then we
check if the node from current path on same level is same. If so we save the
level. In the end we shorten current path to length equal to highest changed
level and we can start with next iteration.

\subsection{Developing without creating child}

This enhancement saves memory. Game tree is largest when root is solved.  In
this state there is a lot of nodes which are undeveloped. \obr{ 11) hotovy strom se
spoustou listu} We could reduce memory requirements if we don't create part of
them. 

When we are developing node we create and evaluate all sons in order to get
proof and disproof numbers, as usually. We destruct these sons after. When we
find that most proving node is a son of the same node which was developed this
way we create all its sons again and now let them stay in partial game tree.
Every node is developed at most twice so time requirements rise at most twice.
\obr{12) 11) po tomhle } 

We can save maybe even more memory when we use some counter in nodes
and save only developed sons or all sons if their percentage or number is 
above some threshold.

\subsection{Cache }

There is a possibility to use cache when we use dp-pn-search \ref{dfpn}.
We can use cache also when we use ordinary pn-search but we must
do some modifications in this case: 

\begin{enumerate}
\item Do not delete from cache node is on current path.
\item Stop updateAncestor when ancestor is deleted.
\item When some child is missing create it again.
\end{enumerate}

Both dp-pn-search and modified pn-search have problem that when cache is really small (slightly
more then maximal current path length). In this case it could happen that algorithm
never solves the root. It could happen that root has
two sons and algorithm runs for a while in subtree of one and after in second
subtree disposing all from first and then returns back to first to improve
it but it can't because it must create nodes again (hence deleting nodes from the
second subtree) and the situation repeats ad infinitum. 
Hopefully this doesn't happen with big cache.

We use \TODO{nazev-prosim poradte.. srustajici retizky} cache. When we are adding new node to
cache we check in place with number equal to nodes hash. If this place is free
we insert the node here. If not we check if some of next $c$ places is free and
if so we insert the node on first free place. If none of the places is free we
must delete some of this $c$ nodes. When we are searching for some node we
start checking on place with number equal to nodes hash and continue until we
found it or we found free place or we search $c$ places. In last two cases the
node isn't in cache. Last thing to specify is how to decide which node delete
from cache. We cannot delete node from current path and sons of node which is
just developing. We can set other rules which nodes cannot be deleted or aren't
preferred for deleting. When we set too many rules it can happen that there in
no node which can be deleted and in this case we halt whole program. 

Our experience is that it happens early or never.

\subsubsection{Example of rules: \TODO{ne example,delsi}}
\begin{itemize}
\item Don't delete node on current path and sons of node which we are developing.
\item Do not prefer nodes with value \value{true} or \value{false}.
\item Do not prefer nodes with many parents - these are more important than nodes
      with few parents.
\item Prefer nodes which aren't developed - we don't lose any important information
      when we delete these. 
\end{itemize}

\section{Clique game - graph representation }

We need to save game position during searching the game tree. In other words we
need to save which edges are claimed and which color they have. In
this section we introduce few suitable solutions and their properties.

Generally we use adjacency matrix but there are several ways how to represent
it in memory. Adjacency matrix is preferred over other structure (adjacency list etc.) because
most position we search are in depth where graph is \TODO{hustý}.

\subsection{Basic representation }
The adjacency matrix is straightforwardly represented as two 
dimensional array $N \times N$ of numbers where value on the position
$(i,j)$ is zero if edge $(i,j)$ is free or color number of edge if some player claimed it.

This representation is unnecessarily large. For saving space we need just two
bit information for every edge and we can use the symmetry of the adjacency
matrix. 

\subsection{Triangle in a~line }

Here we store the lower triangular part of the adjacency matrix in a line of
bits. Supposing that size of int is at least ${N \choose 2} * 2$ we can use
single int. If this doesn't hold we can use more ints instead. For each edge
$(i,j)$ we set two bits. First, if the edge is free we set the bit at position
$((u*(u+1))/2+v)*2$. Second, to store the color of the edge we set the bit at
position $((u*(u+1))/2+v)*2+1$.

\subsubsection{ Two small improvements }

There is still space for improvement. Imagine that every two bits in line form
a binary number \--- it can be 0,1,2 never 3. So, structure from the last
paragraph is a~number in base three. We can convert this number into base two
before storing and convert it back before reading. In this case structure has
size $ 3/4=75\% $ of the original structure. It is the best solution if we care
only about space, but converting between bases is slow.

Faster possibility is to break the number into parts by three edges (stored
in six bits) and convert these short numbers into binary, which can be done
quickly using translation table. This structure has size $5/6 = 83\% $. Similar
procedure can be used for parts by $n$ edges. If $n=5$ structure has size 
$8/10=80\%$. For bigger $n$ reduction in space complexity can be bigger
but at the cost of growth of translation tables.

\subsection{ Adjacency matrix in line}

Hear we store whole adjacency matrix of red edges in one line of bits and blue
in second line of bits. Again, we suppose that we have int with size at least
$N^2$. Information if edge has red color is stored at position $i*N+j$ and
also at position $j*N+i$ in red line, and correspondingly with blue edges and
blue line.

This structure is slightly more than twice larger than previous, but it has
several advantages in speed of operations, and it's easier for programing which
is also important. Only this variant was implemented in our software.

\subsubsection{ Trick: Find common neighbours }

We often need to know which nodes $v$ have edges with red color to nodes $i$
and $j$ simultaneously. If we use basic representation there is one way how to
find it. We look if red edges $(i,v)$ and $(j,v)$ exist for every node $v$ ($v
\ne i \; \& \; v \ne j$). If we use last described structure, there is a~trick.
We can use bit mask and bit shift to get $i$-th and $j$-th row of matrix, that
is bits on position $i*N \ldots (i\!+\!1)\!*\!N\!-\!1$ and $j*N \ldots
(j\!+\!1)\!*\!N\!-\!1 $. Then we just AND these rows. Every nonzero position
indicates such $v$.

\subsubsection{ Find claimed winning set } 

When red player claims edge $(i,j)$ we want to know if he wins (determine if he
created red $K_4$). Straight solution is to test every edge between every set
of four nodes which contain $i$ and $j$.

We can also use a~trick. We can find all nodes which can be in winning set very
quickly. Then, we test if in this set are distinct nodes $u$ and $v$ such that
red edge $(u,v)$ exist.


\subsubsection{ Find free winning sets }

It can by useful to find if there is a winning set which opponent haven't
claimed yet and how many such sets exist. We can stop searching game tree if
there are no free winning sets and we can prefer to turn to edges which is
contained in the most free winning sets.

Free winning sets can by found in a similar way to claimed winning sets. We use
complement of opponents line instead players line.

\subsubsection{ Find threat }

We want to know if player who just played created a~threat. If he created one
thread his opponent has only one possible turn and if he created more then one
thread he wins. 

Threat is a state where there are five edges with same color and one free edge
between four nodes. Suppose that last turn was into $(i,j)$. There are two
types of threats.
 
\begin{enumerate} 
	\item Free edge begins on node $i$ or $j$ (without loose of generality $i$). 
	\item Free edge doesn't contains any of them.  
\end{enumerate}

In first case we use the trick to find all nodes $u$ which are neighbours of both
$i$ and $j$, and again to find all nodes $v$ which are neighbours of
both $j$ and $u$. If there is such $u$ and $v$ we check if there is free edge
$(i,v)$. If yes then $i$,$j$,$u$ and $v$ is thread.  \obr{ 13a) hrozba}

Second case is similar to finding claim winning set. We just check if there is free
edge between $(u,v)$ instead of red.  \obr{ 13b) hrozba}

\subsubsection{ Compute degree }

We can compute color degree of any vertex $v$ very quickly. We precompute table of size
$2^N$ containing all answers for degree questions. Then we just indexing this table by
row $v$. 

\section{ Games on full graph - Norm } \label{norm} 

Many isomorph game position and their subtrees are traversed during ordinary
PN-search multiple times. If we want to use advantage of hight symmetry of the
game we need to join these positions and reduce number of subtrees we must
search.

In an ideal case we want to search subtree of each graph in isomorph class only
once. When we are creating new graph $G$ we compute \TODO{spravny nazev}
representative graph $H$ of isomorph class where $G$ belongs, and then use $H$
instead of $G$. Enhancement transpositions into DAG \ref{DAG} arranges that
each subtree of two isomorph graphs will be searched at most once.

The difficulty of this lies in computing representative graph, which is hard problem.
 \TODO{np-těžký nebo np-úplný? jestě těžší.. citace}
So we will compute almost representative graph instead. We will call
them norm graphs.

\newtheorem*{norm}{Definition}	
\begin{norm}
{\sl Norm function} is a~function that for
every graph $G$ returns graph $N$ such that $N$ is isomorph with $G$. We we will call
$N$ is a~{\sl norm graph}.
\end{norm}

We want to find norm function which can by computing fast and
which is effective. Here efficiency means if we apply the function on
isomorph class of graph we receive as small set of norm graphs as is possible,
or in other words there are only few distinct norm graphs which are isomorph.

We introduce few such function and compare their effectiveness and speed on
graphs with five nodes and $k$ edges. See table. Column number in row $k$ says
how many norm graphs after turn $k$ exist, for simplicity we don't care if
there is claimed winning set. Column time says average time of computing in
milliseconds in our software on PC \footnote{\TODO{porametry celera}}.
 

\TODO{tabulka+pridat radek sum}
[1,10,90,360,1260,2520,4200,4200,3150,1260,252] identita
[1,1,2,9,31,60,96,96,69,28,22] norm1
[1,1,2,7,24,36,63,63,55,28,22] norm2 -blbe predelat dvoji sum
[1,1,2,6,17,30,47,47,37,16,6] pefect

\subsubsection{Identity}
Trivial to compute but ineffective.

\subsubsection{Sort by degree}
We can compute degree of each node and sort nodes by degree. Basic improvement
is to count every red edges $M$ times where $M$ is number of edges in full
graph. This improvement decreases number of norm graphs which are isomorph.

\subsubsection{Sort by adjacent degree} Previous function fails in case shown
in \obr{14) 2 grafy co by se driv znormovaly a ted ne}. It helps to consider
degree of adjacent nodes. We define functions $f()$ and $g()$ on nodes. 
\obr{14)obrazek popis: This two graphs was both norm but isomorph before
improvement but now norm function must return same graph.}

\begin{eqnarray*} 
f(v) = \mbox{blue}\Delta (v) * M^2 + \mbox{red}\Delta (v) * M^5.
\end{eqnarray*}

\begin{eqnarray*} 
g(v) = f(v) + \sum_{(u,v) \mbox{is red}}{\frac{f(u)}{M}} + \sum_{(u,v) \mbox{is blue}}{\frac{f(u)}{M^2}}
\end{eqnarray*}

Then norm function just sorts edges by $g()$.

This function is asymptoticly slower than previous, but still polynomial
and more effective.

\subsubsection{Ideal case} 
We assign representative graph of isomorph class.
This is the most effective solution but hard to compute as we noted before.
This function wasn't implemented so we can't measure its time.
Efficiency was computed this way: We use previous norm function
and for every two graphs with same values $g()$ we straightforwardly test 
if they are isomorph. 

\section{ Games on full graph - hash } 




