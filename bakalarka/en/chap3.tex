\chapter{Further theoretical improvements}

In this chapter we examine various improvements to pn-search, clique
game and general positional games on complete graph. All results
were independently discovered by the author of this thesis.

\section{Other enhancements of PN-search} 
Here we describe several enhancements
and also show some solutions for problems which arise when we combine several
\sec{PN-search} enhancements together. 

\subsection{Ordering of sons} \label{ord}

The ordering of sons of some node could be important. When there are two or more
sons with a~minimal (dis)proof number we usually choose the leftmost but there could
be better variant so we order sons by some heuristic.

\subsection{Deleting solved sons}

When we disproof a~son of \node{or} node (and symmetrically when we proof a~son
of \node{and} node) we do not solve the node but we can we can delete the edge
between them. The node will have less sons so the sum or max or min of numbers
could be counted faster and update ancestors will be also faster. A~bigger
improvement is when using the \sec{weak PN-search}. \sec{Weak PN-search} depends on
the branching factor, which decreases.

\subsection{Deleting parallel edges } \label{multi-edges} 

When we use \sec{transposition to DAG} multiple parallel edges may be
created, for example on the beginning of clique game. The root has $N \choose
2$ sons. Sons, all of them represent $K_N$ with one edge colored. When we
normalise them they will all be the same, and so we have $N \choose 2$ parallel
edge from the root to the same node. For a~reason similar to the previous
subsection, it is a~good idea to change parallel edges to a~single edge.

\subsection{Developing a~node without creating sons} \label{lastenh}

This enhancement saves memory. The partial game tree is largest when the root is
solved. In this state there is a~lot of nodes which are undeveloped. \obr{ 11) vysvetlivky
hotovy strom se spoustou listu} We could reduce memory requirements if we do not
create part of them. 

When we are developing a~node $n$ we create and evaluate all $n$ sons in order to
get proof and disproof numbers, as usually. Afterwards, we delete these sons. When
we find that the  most proving node is a~son of $n$ we create all sons of $n$ again and
now let them stay in the partial game tree. Every node is created at most twice so
time requirements rise at most twice. \obr{12) 11) po tomhle} 

Perhaps we can save even more memory when we use some counter in nodes
and save only the developed sons or all the sons if the percentage or number 
of developed sons is above some threshold.

\subsection{Cache}

There is a~possibility to use a node cache, instead of the representation of the whole
tree, when we use \sec{DF-PN-search}.
We can use a cache also when we use ordinary \sec{PN-search} but we must
do some modifications in this case: 

\begin{enumerate}
\item Do not delete nodes on the current path.
\item Stop \fce{updateAncestors()} when an ancestor is missing from the cache.
\item When a son is missing in the cache, create it again.
\end{enumerate}

Both \sec{DF-PN-search} and modified \sec{PN-search} perform poorly when the
cache is really small (slightly more then the maximal current path length). In
this case it could happen that algorithm never solves the root. It could happen
that the root has two sons $s_1$ and $s_2$  and algorithm runs for a~while in
the subtree of $s_1$ and then in the subtree of $s_2$, deleting all the nodes
of the subtree of $s_1$ from the cache. When it returns back below $s_1$ to
improve it, it cannot because it must create the nodes again (hence deleting
the nodes from subtree of $s_2$) and the situation repeats ad infinitum.
Hopefully this does not happen with a big cache.

Here we specify how to decide which node to delete from the cache. We cannot
delete a node from the current path and sons of a node which is just being
developed. We can set other rules about which nodes cannot be deleted or set
priorities for deleting. This can have significant impact on the performance of
the algorithm as we might delete results which took a long time to compute.

When we set too many rules it can happen that there is no node which can be
deleted and in this case we halt whole program. In our experience with our
solver is that it happens very soon after program starts or never.

\subsubsection{List of possible rules for deleting from cache:}
\begin{itemize}
\item Do not delete a node on the current path and sons of a node which is being developed.
\item Do not prefer nodes with value \val{true} or \val{false} -- they are 
      more important than the nodes with value \val{unknown} (but we need a really big
      cache for it).
\item Do not prefer nodes with many parents -- these are more important than nodes
      with few parents.
\item Prefer nodes which are not developed -- we do not lose any important information
      when we delete these. 
\end{itemize}

\subsection{Variants of \fce{updateAncestors()}}

There are three enhancements which influence the function \fce{updateAncestors()}. 

The first is using \sec{last changed node} to shorten the path. 

The second is \sec{DF-PN-search} (even if there is no function named \fce{updateAncestors()} and it is
updating lazily).

The third is \sec{transposition to DAG}. We found no article
considering changes in \fce{updateAncestors()} when node has more then one parent.

When we use the ordinary \fce{updateAncestors()} together with
\sec{transposition to DAG} and one of the other two enhancements,
\TODO{pravopis} it could happen that some numbers are changed on the current path
above an unchanged node. For an example, see Figure \obr{10)} root
was changed and its son on current path no. It could also happen that the most
proving node found by the algorithm could be different than most proving node
by the definition, see Figure \obr{15)}. This is not as big problem as it seems to be -- the
algorithm is still correct. An unusual situation can happen when we also use
\sec{deleting solved subtrees} -- as a~Czech proverb warns we could ``saw off
a~branch which is holding us''
 
To avoid it we must use new \fce{updateAncestors()} which updates \emph{all}
paths up to root, and better \fce{selectMostProving()} as we show in the next
subsection. 

\subsubsection{Simple updateAncestors()}

A~simple way how to define such \fce{updateAncestors()} is recursively. The
function updates a node and if it changed either proof or disproof
number, we call the function recursively on parents. It could happen that
some nodes are updated many times. So its parents are also updated many times
and the number of updates is growing with the number of possible paths.

\subsubsection{Level updateAncestors()}

A~better solution is to update each node on some level and create a~list of
parents which need to be updated. After we dispose of duplicates in the list,
we do the same on a~higher level. 

\subsubsection{One-visit updateAncestors()}

Another solution is to store more information, namely in which iteration of
\sec{PN-search} was the node last updated. In this case we can update
recursively and stop when we would update some node twice. It could happen
that this algorithm does not count right values but we can still use it as an effective 
approximation.

\subsection{Variants of \fce{selectMostProving()}}

As you can see in Figure \obr{10}, it could happen that some numbers are changed on
the current path above an unchanged node. So the observation from Section \sec{last changed
node} (\ref{last}) does not hold and it can happen that the algorithm select a most
proving node wrongly. There is an~example in Figure \obr{15}. We found several solutions.

\subsubsection{Start from the root}

The simplest solution is to use neither enhancement \sec{last changed node}
nor \sec{DF-PN-search}. In each iteration we start the search from the root.

\subsubsection{Start at the highest changed level}

We can use this solution when we have the current path in a stack, which usually
happens when we use recursion. Function \fce{updateAncestors()} will compute the level of
the highest changed node. We pop enough nodes from the stack so that the stack has
the length equal to the level of the highest changed node. In the next iteration, we can
start searching from the node on top of the stack. This way
we find the same most proving node as when we are searching from the root. We can
use the same argument as in Section \sec{last changed node} (\ref{last}) to prove it.

\subsubsection{Remember the current path}

Another variant is to start at the level of the highest changed node of the current path.
We hold the whole current path in an array. In every
node which is updated we test if it is in the current path which can be done
quickly. The level of the updated node can be found as a~number of positions
claimed by all players or we can save this information in the node.

%\section{\TODO{veta o df}}

\section{Graph representation for games on complete graph} 

We need to store a game state during searching the partial game tree. In other words we
need to save which edges are claimed and which color they have. In
this section, we introduce several suitable solutions and their properties.

Generally, we use an adjacency matrix but there are several ways how to
represent it in memory. An adjacency matrix is preferred over other structures
(adjacency list etc.) because most game states we search are in depth where the
colored subgraph is dense.

\subsection{Basic representation }
The adjacency matrix is straightforwardly represented as two 
dimensional array $N \times N$ of numbers where the value on the position
$(i,j)$ is zero if edge $(i,j)$ is free, or color number of the player who claimed it.

\subsection{Triangle in a~line }

The previous representation is unnecessarily large. For saving space we need just
two-bit information for every edge and we can use the symmetry of the adjacency
matrix. 

In this solution, we store the lower triangular part of the adjacency matrix in a~line of
bits. Supposing that the size of the computer word is at least ${N
\choose 2} \cdot 2$ bits we can use a single word. If this does not hold we can use more
words instead. For each edge $(i,j)$, we set two bits. First, if the edge is free
we set the bit at the position $({u \choose 2}+v)\cdot2$. Second, to store the color of
the edge we set the bit at the position $({u \choose 2}+v)\cdot2+1$.

\subsubsection{Two small improvements}

There is still space for improvement. Imagine that every two bits in a line form
a binary number -- it can be 0, 1 or 2, never 3. So, the structure from the last
paragraph is a~number in base three. We can convert this number into base two
before storing and convert it back before reading. In this case the structure has
size $ \log_2(3)/\log_2(4) \doteq 79\% $ of the original structure. It is the best solution if we care
only about space, but converting between bases is slow.

A faster possibility is to break the number into parts by three edges (stored
in six bits) and convert these short numbers into binary, which can be done
quickly using a translation table. This structure has the size $5/6 \doteq 83\%
$ of the original. A similar procedure can be used for parts of $n$ edges. If
$n=5$ the structure has size $8/10=80\%$. For a bigger $n$ the reduction in space
usage is not much better.

\subsection{Adjacency matrix in a~line}

In our algorithm, we store the whole adjacency matrix of red edges in one line of bits and blue 
in the second line of bits. Again, we suppose that we have word of size at least
$N^2$. Information if an edge has red color is stored at position $i \cdot N+j$ and
also at position $j \cdot N+i$ in the red line, and correspondingly with blue edges and
the blue line.

This structure is slightly more than twice as large than \sec{triangle in a~line}, but it has
several advantages in speed of operations, and it is easier for programming which
is also important. 

\subsubsection{A trick: Detect common neighbours}

We often need to know which nodes $v$ have red (resp. blue) edges to nodes $i$ and $j$
simultaneously. If we use basic representation there is one obvious way to find it.
We look if red edges $(i,v)$ and $(j,v)$ exist for every node $v$ ($v \ne i \;
\& \; v \ne j$). 

If we use \sec{adjacency matrix in line}, there is a better trick. We can
use binary mask and binary shift to get $i$-th and $j$-th row of the matrix, that is, bits
on positions $i \cdot N, \ldots, (i\!+\!1)\! \cdot \!N\!-\!1$ and $j \cdot N, \ldots,
(j\!+\!1)\!\cdot\!N\!-\!1 $. Then we just apply logical \fce{and} on these rows. Every nonzero bit
indicates such vertex $v$.

\subsubsection{Finding a claimed winning set} 

When the red player claims the edge $(i,j)$ we want to know if he wins (determine if he
created red $K_4$). A straightforward solution is to test every edge between every set
of four nodes which contain $i$ and $j$. 

We can also use the trick from previous subsection to do it faster. We find all
the nodes which have red edges to both $i$ and $j$. Then we test if in this
set are distinct nodes $u$ and $v$ such that red edge $(u,v)$ exists. This is illustrated
in Figure \obr{13z)}.


\subsubsection{Finding free winning sets} \label{freeK4}

It can be useful to find out if there is a~free winning set 
and how many such sets exist. We can stop searching the partial game tree if
there are no free winning sets and we can prefer to turn to edges which are
contained in most free winning sets.

Free winning sets can be found in a~similar way as the claimed winning sets. We just 
use complement of opponents line instead of players line. 

\subsubsection{Finding threats}

We want to know if the player who just played created a~threat. If he created one
threat his opponent has only one possible turn and if he created several
threat disjoint on the other two vertices, he wins. 

Threat is a~state where there are five edges with the same color and one free edge
between four nodes. Suppose that the last turn was to $(i,j)$. There are two
types of threats.
 
\begin{enumerate} 
	\item The free edge begins on node $i$ or $j$ (without loss of generality $i$). 
	\item The free edge does not contains any of them.  
\end{enumerate}

In the first case we use the trick to find all nodes $u$ which are neighbours of both
$i$ and $j$, and again to find all nodes $v$ which are neighbours of
both $j$ and $u$. If there are such $u$ and $v$, we check if there is free edge
$(i,v)$. If yes then $\{i,j,u,v\}$ is a threat. See Figure \obr{ 13a) hrozba}.

The second case is similar to finding a claimed winning set. We just check if there is a free
edge between $u,v$ instead of a red edge. See Figure \obr{ 13b) hrozba}.

\subsubsection{Computing vertex degree}

We can compute the color degree of any vertex $v$ very quickly. We precompute table of size
$2^N$ containing all the answers for degree queries. Then we just index this table by
the $v$-th row of the adjacency matrix. 

\section{Normalizing function for games on a complete graph} \label{norm} 

Many nodes with isomorphic graphs and their subtrees are traversed during an ordinary
\sec{PN-search}. If we want to use the advantage of a high symmetry of the
game we need to join these positions and hence reduce the  number of subtrees we must
search.

In each node is a~clique game state which is an undirected complete graph with
edges colored by three colors (red, blue, free).
In an~ideal case we would search the subtree of isomorphism class
only once. When we are creating a new graph $G$ we compute a canonical
representation $H$ of the isomorphic class where $G$ belongs, and then use $H$ instead
of $G$. The enhancement \sec{transpositions to DAG} causes that the
subtree of two isomorphic graphs will be searched at most once.

However, in computing a canonical representative of a graph is a difficult
problem (see \cite{canonical}). So we compute an almost representative graph
instead. We call them a~normalise graphs.

\newtheorem*{norm}{Definition}	
\begin{norm}
{\sl A~normalising function} is a~function that for
every graph $G$ returns a~graph $H$ such that $H$ is isomorph with $G$. We we will call
$H$ a~{\sl normalise graph}.
\end{norm}

We want to find a normalise function which can be computed fast and which is
effective. Here efficiency means if we apply the function to every graph of an
isomorphism class we receive as small set of normalised graphs as possible, or
in other words there are only few distinct normalised graphs which are
isomorphic.

We introduce several such functions and compare their effectiveness and speed
on graphs\footnote{undirected 3-colored complete graphs} with five nodes and
$k$ edges colored as in clique game. See Table \ref{statsNorm}. 

\begin{table}
\centering
\begin{tabular}{c|c|c|c|c|c|c|c|c|c|c|c|c}
$k$&1&2&3&4&5&6&7&8&9&10\\
\hline
\textbf{1}&1&10&90&360&1260&2520&4200&4200&3150&1260&252\\
\textbf{2}&1&1&2&9&31&60&96&96&69&28&22\\
\textbf{3}&1&1&2&7&24&36&63&63&55&28&22\\
\textbf{4}&1&1&2&6&17&30&47&47&37&16&6\\
\end{tabular}
\caption{Efficiency of normalise functions}
\label{statsNorm}
\end{table}

\subsubsection{1 Identity}
Trivial to compute but ineffective.

\subsubsection{2 Sorting by vertex degree} \label{norm1}

We can compute the degree of each node and sort the nodes by degree. A basic improvement
is to count every red edge $M$ times where $M$ is the number of edges in a complete
graph. This improvement decreases the number of normalised graphs which are isomorphic.

\subsubsection{3 Sorting by vertex and neighbours degrees} \label{norm2}

Previous function fails in case shown in Figure \obr{14) }. In case of a tie, tt helps to
consider the degree of adjacent nodes. We define functions $f$ and $g$ on the  nodes.
\obr{14)obrazek popis: This two graphs was both normalise but isomorph before
improvement but now normalise function must return same graph.}

\begin{eqnarray*} 
f(v) = \mbox{blueDeg}(v) + \mbox{redDeg}(v) \cdot M^3.
\end{eqnarray*}

\begin{eqnarray*} 
g(v) = f(v) \cdot M^2 + \sum_{(u,v) \mbox{is red}}{f(u) \cdot M} + \sum_{(u,v) \mbox{is blue}}{f(u)}.
\end{eqnarray*}

Then the normalising function just sorts the vertices by $g$.

This function is asymptoticly slower than the previous one, but still polynomial
and more effective.

\subsubsection{4 Ideal case} 
We assign the canonical representation of graph.
This is the most effective solution but hard to compute as we noted before.
This function was not implemented so we cannot measure its speed.
Efficiency was computed in the following way: We use the previous normalising function
and for every two graphs with the same values of $g$ we straightforwardly test 
if they are isomorphic. 

