%%% Hlavní soubor. Zde se definují základní parametry a odkazuje se na ostatní části. %%%

%% Verze pro jednostranný tisk:
% Okraje: levý 40mm, pravý 25mm, horní a dolní 25mm
% (ale pozor, LaTeX si sám přidává 1in)
% \documentclass[12pt,a4paper]{report}
% \setlength\textwidth{145mm}
% \setlength\textheight{247mm}
% \setlength\oddsidemargin{15mm}
% \setlength\evensidemargin{15mm}
% \setlength\topmargin{0mm}
% \setlength\headsep{0mm}
% \setlength\headheight{0mm}
%% \openright zařídí, aby následující text začínal na pravé straně knihy
% \let\openright=\clearpage

% Pokud tiskneme oboustranně:
 \documentclass[12pt,a4paper,twoside,openright]{report}
 \setlength\textwidth{145mm}
 \setlength\textheight{247mm}
 \setlength\oddsidemargin{15mm}
 \setlength\evensidemargin{0mm}
 \setlength\topmargin{0mm}
 \setlength\headsep{0mm}
 \setlength\headheight{0mm}
 \let\openright=\cleardoublepage

%% Použité kódování znaků: obvykle latin2, cp1250 nebo utf8:
\usepackage[utf8]{inputenc}

%% Ostatní balíčky
\usepackage{graphicx}
\usepackage{amsthm}
\usepackage{color}

%% Balíček hyperref, kterým jdou vyrábět klikací odkazy v PDF,
%% ale hlavně ho používáme k uložení metadat do PDF (včetně obsahu).
%% POZOR, nezapomeňte vyplnit jméno práce a autora.
%\usepackage[ps2pdf,unicode]{hyperref}   % Musí být za všemi ostatními balíčky
%\hypersetup{pdftitle=: Algorithms for solving strong positional games with high symmetry}
%\hypersetup{pdfauthor= Jitka Novotná}

%%% Drobné úpravy stylu

% Tato makra přesvědčují mírně ošklivým trikem LaTeX, aby hlavičky kapitol
% sázel příčetněji a nevynechával nad nimi spoustu místa. Směle ignorujte.
\makeatletter
\def\@makechapterhead#1{
  {\parindent \z@ \raggedright \normalfont
   \Huge\bfseries \thechapter. #1
   \par\nobreak
   \vskip 20\p@
}}
\def\@makeschapterhead#1{
  {\parindent \z@ \raggedright \normalfont
   \Huge\bfseries #1
   \par\nobreak
   \vskip 20\p@
}}
\makeatother

% Toto makro definuje kapitolu, která není očíslovaná, ale je uvedena v obsahu.
\def\chapwithtoc#1{
\chapter*{#1}
\addcontentsline{toc}{chapter}{#1}
}

% Me makra.
\newenvironment{itemize*}%
{\begin{itemize}%
\setlength{\itemsep}{30pt}%
\setlength{\parskip}{-30pt}}%
{\end{itemize}}

\def\TODO#1{
\texttt{\textcolor{red}{#1}}
}
\let \node \textsf
\let \val \textsf
\let \fce \textsl
\let \sec \textsc
\let \com \textsc

\begin{document}

% Trochu volnější nastavení dělení slov, než je default.
\lefthyphenmin=2
\righthyphenmin=2

%%% Titulní strana práce
\pagestyle{empty}
\begin{center}

\large

Charles University in Prague

\medskip

Faculty of Mathematics and Physics

\vfill

{\bf\Large BACHELOR THESIS}

\vfill

\centerline{\mbox{\includegraphics[width=60mm]{../img/logo.eps}}}

\vfill
\vspace{5mm}

{\LARGE Jitka Novotná}

\vspace{15mm}

% Název práce přesně podle zadání
{\LARGE\bfseries Algorithms for solving strong positional games with high symmetry}

\vfill

% Název katedry nebo ústavu, kde byla práce oficiálně zadána
% (dle Organizační struktury MFF UK)
Computer Science Institute of Charles University
\vfill

\begin{tabular}{rl}

Supervisor of the bachelor thesis: & Tomáš Valla \\
\noalign{\vspace{2mm}}
Study programme: & Informatika \\
\noalign{\vspace{2mm}}
Specialization: & Obecná informatika \\
\end{tabular}

\vfill

% Zde doplňte rok
Prague 2012

\end{center}

\newpage

%%% Následuje vevázaný list -- kopie podepsaného "Zadání bakalářské práce".
%%% Toto zadání NENÍ součástí elektronické verze práce, nescanovat.

%%% Na tomto místě mohou být napsána případná poděkování (vedoucímu práce,
%%% konzultantovi, tomu, kdo zapůjčil software, literaturu apod.)

\openright

\noindent
I want to thank a~lot of people who helped me with this thesis. First I want to
thank Jan Kulveit, Tomáš Gavenčiak and Martin Mareš - I'm grateful for their
time, support, good advice and also proofreading. I thank Pavel Veselý, Pavel
Dvořák and Roman Smrž for helpful discussions and aid with debugging my code.
Finally I thank my advisor, TV, for an unrelenting supply of fresh ideas what
can be extended and improved.

\newpage

%%% Strana s čestným prohlášením k bakalářské práci

\vglue 0pt plus 1fill

\noindent
I declare that I carried out this bachelor thesis independently, and only with the cited
sources, literature and other professional sources.

\medskip\noindent
I understand that my work relates to the rights and obligations under the Act No.
121/2000 Coll., the Copyright Act, as amended, in particular the fact that the Charles
University in Prague has the right to conclude a~license agreement on the use of this
work as a~school work pursuant to Section 60 paragraph 1 of the Copyright Act.

\vspace{10mm}

\hbox{\hbox to 0.5\hsize{%
In ........ date ............
\hss}\hbox to 0.5\hsize{%
signature of the author
\hss}}

\vspace{20mm}
\newpage

%%% Povinná informační strana bakalářské práce

\vbox to 0.5\vsize{
\setlength\parindent{0mm}
\setlength\parskip{5mm}

Název práce:
Algorithms for solving strong positional games with high symmetry
% přesně dle zadání

Autor:
Jitka Novotná

%Katedra:  % Případně Ústav:
Ústav:
% Název katedry či ústavu, kde byla práce oficiálně zadána
Informatický ústav Univerzity Karlovy
% dle Organizační struktury MFF UK

Vedoucí bakalářské práce:
RNDr. Tomáš Valla, Informatický ústav Univerzity Karlovy
% Jméno a příjmení s tituly, pracoviště
% dle Organizační struktury MFF UK, případně plný název pracoviště mimo MFF UK

Abstrakt: {\parskip=0pt
% abstrakt v rozsahu 80-200 slov; nejedná se však o opis zadání bakalářské práce

V této práce představujeme několik algoritmů pro počítačové řešení silných
pozičních her a to především algoritmů založených na algoritmu PN-search.

Zaměřujeme se na hry s vysokou symetrií herního plánu. Spojováním mnoha
izomorfních pozic jsme dosáhli velkého zmenšení části herního
strumu, kterou je potřeba prohledat. 

Představujeme již známa vylepšení a také navrhujeme vlastní. Ověřujeme
jak tyto heuristiky funguji na příkladu klikové hry.

K práci je přiložen software pro řešení silné klikové hry pro $K=4$ a
$N=5\dots8$. Software zvládl vyřešit (6,4) a (7,4)-klikovou hru a dokázal, že
druhý hráč má neprohrávající strategii, což se očekávalo, ale dosud nebylo
potvrzeno.}

Klíčová slova:
% 3 až 5 klíčových slov
PN-search, silné poziční hry, vysoká simetrie, kliková hra

\vss}\nobreak\vbox to 0.49\vsize{
\setlength\parindent{0mm}
\setlength\parskip{5mm}

Title:
Algorithms for solving strong positional games with high symmetry
% přesný překlad názvu práce v angličtině

Author:
Jitka Novotná

Department:
Computer Science Institute of Charles University
% Název katedry či ústavu, kde byla práce oficiálně zadána
% dle Organizační struktury MFF UK v angličtině

Supervisor:
RNDr. Tomáš Valla, Informatický ústav Univerzity Karlovy 
% dle Organizační struktury MFF UK, případně plný název pracoviště
% mimo MFF UK v angličtině

Abstract: {\parskip=0pt
% abstrakt v rozsahu 80-200 slov v angličtině; nejedná se však o překlad
% zadání bakalářské práce

In this thesis, we analyse several algorithms for solving strong positional games, 
mostly based on PN-search.

We focus on games with high symmetry of the game plan, where it is possible to 
substantially reduce the partial game tree by joining isomorph positions.

We review several known enhancements of PN-search and also propose some of our own design. 
We measure the effect of the enhancements on the clique game.

A part of the thesis is a software solver for clique game for $K=4$ and $N=5\dots8$. We were able 
to solve (6,4) and (7,4) clique games and prove that the second player has a draw 
strategy, which was expected but not shown previously.
}

Keywords:
% 3 až 5 klíčových slov v angličtině
PN-search, strong positional games, high symmetry, clique game

\vss}

\newpage

%%% Strana s automaticky generovaným obsahem bakalářské práce. U matematických
%%% prací je přípustné, aby seznam tabulek a zkratek, existují-li, byl umístěn
%%% na začátku práce, místo na jejím konci.

\openright
\pagestyle{plain}
\setcounter{page}{1}
\tableofcontents

%%% Jednotlivé kapitoly práce jsou pro přehlednost uloženy v samostatných souborech
\chapter*{Introduction}
\addcontentsline{toc}{chapter}{Introduction}

This thesis describes an algorithm for solving strong positional games
with high symmetry. We describe the most important algorithm for solving 
combinatorial games -- PN-search and its various enhancements, which change
it so much, that they can be considered new algorithms. 

\medskip

PN-search was introduced by Allis in 1994\cite{allis}. It has become well a know algorithm
and a basis for a~lot of further work. It was successfully applied for a~wide class
of games, and many enhancements were invented. We review some of them and how they can be useful
in case of a~game with high symmetry. We consider which  enhancements
can be used together and what problems can arise with these combinations.
We also introduce some enhancements which we discovered independently.

\subsubsection{Outline}

In the first chapter, we informally introduce some combinatorial
game terminology. We suppose the reader has general knowledge about graph theory, 
see Diestel~\cite{ramsey}. In the first chapter we also define clique game and give
solutions of some cases of a~clique game.

Known algorithms and their enhancements are introduced in the second chapter. Usually,
algorithms are designed for a general combinatorial game. We apply them to
strong positional games, which simplifies the descriptions. We start with a~game 
solving meta algorithm. Some specific algorithms are described after. We choose PN-search to
implement and to be described in detail.

Theoretical results, which we found independently, are described in the third chapter.
In subsections~\ref{ord} to~\ref{lastenh2} are enhancements of PN-search. Then we introduce some rules for
deleting from the cache which are designed for strong positional games. We consider
a~problem which can occur when we combine some enhancements, and we
propose a~solution for the problem. In the end of the chapter, we focus on
a~graph of a~clique game. We describe several ways of storing
the game state and also how to normalize them.

The last chapter introduces our software and presents experimental results.
First, we describe the partial game tree created by our solver during solving
(6,4) and (7,4)-clique games. Then we compare some mentioned enhancements. In the 
end of the chapter, we describe the software implementation in more detail.

\chapter{Games}

\TODO{node, 
game position
game tree=DAG 
free winning set,
symmetric game 
finite
hrozba
definice turn-to co bezne, to co ja,
free edge/claim 
grafem myslim dvoubarevny neorientovany
ze druhy nema vyhravajici strategii
definice strategie
win,draw position
AND OR - bitove operace
} 
\section{ Definition }

\subsection{ Two players sequential game}

\subsection{ Positional game}
\TODO{ala piskvorky}

\subsubsection{ Strong game }

\subsubsection{ Maker-Breaker games }

\subsection{ Other property }
\TODO{ Hight symmetry, strong branching factor}

\section{ Clique game }

\subsection{ Rules }

\subsection{ Ramsey theorem }

\subsection{ Solution for some N, K }

\subsection{ Space and time complexity }
\TODO{
neresim slozitost ale velikost stromu 
udelat bubu jak je ten problem tezky
rict ze nema smysl resit asimptotickou slozitost fci v N ze na konstantach a heuristikach zalezi vic
}

 

\section{ Other games }
\TODO{treba ty ctverecky, mb varianta klikvo hry vysledky, citace, reversni varianta}


\chapter{Algorithms and enhancements}

\section{Overview of algorithms}

\subsection{Meta algorithm}

We will describe a meta algorithm how to prove that two-player finite
positional game is determined (that there is a winning strategy for the first
player or draw strategy for the second).

All game progress can be represented as a rooted tree. Each game position is a
node and if someone can turn from one position to another, there is an edge
between the corresponding nodes. We will not try to build a complete game tree,
representing every possible position in the game because it would be too large
to be represented in memory. Instead, we will incrementally  build partial game 
tree starting from empty tree, trying to keep partial game tree as small as
possible.

We will assigned to each node one of three possible values -- \value{true},
\value{false}, \value{unknown}. Node will have value \value{true} when we
proved that there is a winning strategy for the first player and value
\value{false} when we proved that there is a draw strategy for the second
player. If we didn't proved anything node has value \value{unknown}. We will
say that node is solved if it has value \value{true} or \value{false}.

The algorithm starts with one node with value \value{unknown}. In each iteration the
algorithm does one of the two things:

\begin{itemize} 
	\item{Develop node} We choose one leaf node with value \value{unknown}
		and develop it, which means that we create its sons. When 
		a son is created we evaluate it.
	\item{Update node} We choose one developed node with value \value{unknown}
		and look over its sons. We check if we can assign value \value{true} 
		or \value{false} to the node.
\end{itemize}

Algorithm ends when root is solved and so game is determined.

There are two types of nodes. When first player is on turn it is \node{or}
node. When we want to assign value \value{true} for it, it is enough if one of its sons
have value \value{true}. Than, first player has winning strategy which starts by the
turn to the node with value \value{true}. On the opposite when first player hasn't
winning strategy after any turn he hasn't winning strategy at all. So we can
assign value \value{false} when all sons have value \value{false}. If nothing of above holds
the value of the node remains \value{unknown}. Hence value of or node can by found by
operation which is very similar to simple or.

Second type of node is \node{and} node. Second player is on turn. Assigning values
works in similar way, we just use and-like operation instead of or-like.

\obr{obrazek 1) and kolecko} 

\subsection{Depth-first search}
We can use ordinary depth-first search for searching the game tree and solving the root.
We start by creating the root. We developed root and choose one of it sons. We solve the sons
recursively and update the root. We continue solving sons and updating until root is solved.

This algorithm is suitable solution for clique game. Depth of three is $N \choose 2$ so it
doesn't take much memory. This is obviously slow but if we use some of enhancements of pn-search which
are described later this algorithm can find solution for $N=5$ and maybe more.

\subsection{Alpha-beta}

Alpha-beta is well known algorithm for finding good strategy. \TODO{citace}.
We can use it for any node $n$. It searches part of subtree under $n$ and for
nodes in some depth uses rating function and returns rating of $n$. The rating
which alpha-beta returns can by used for decision to which position turn. It
is useful as a heuristic and can be used for example in AI and also in our
problem. We can use depth-first search and when we make decision which sons
solve we rate all its sons by alpha-beta first and then solve in order
determined by ratings.

\subsection{PN-search}

PN-search was found by Allis \TODO{citace}. We will describe immediate evaluation
variant of it which is better suited for our problem, because test if position is winning is
is quick.

PN-search is a best-first search. Main point of this algorithm is to determine
which node is the best. We start with some definitions, which were used in \TODO{allis}. 

\newtheorem*{prove}{Definition}	
\begin{prove}
We will say that we \emph{prove} node if we proved that first player has winning strategy
from it and we can assign \value{true} to it. 
\end{prove}

\newtheorem*{disprove}{Definition}	
\begin{disprove}
We will say that we \emph{disprove} node if we proved that second player has draw strategy
from it and we can assign \value{false} to it. 
\end{disprove}

\newtheorem*{proofSet}{Definition}	
\begin{proofSet}
	For any game tree $T$ a set of nodes with value \value{unknown} $S$ is a~{\sl proof set}
	if proving all nodes within $S$ proves $T$.
\end{proofSet}

\newtheorem*{disproofSet}{Definition}	
\begin{disproofSet}
	For any game tree $T$ a set of nodes with value \value{unknown} $S$ is a~{\sl disproof set}
	if proving all nodes within $S$ disproves $T$.
\end{disproofSet}

\newtheorem*{proofNumber}{Definition} 
\begin{proofNumber}
	For any game tree $T$, the {\sl proof number} of $T$ is defined as the 
	cardinality of the smallest proof set of T.
\end{proofNumber}

\newtheorem*{disproofNumber}{Definition}	
\begin{disproofNumber}
	For any game tree $T$, the {\sl disproof number} of $T$ is defined as the 
	cardinality of the smallest proof set of T.
\end{disproofNumber}

We will show how proof and disproof numbers works and how they can by calculate
on some examples. At picture \obr{2)} you can see tree with proof number.
Nodes with value \value{true} have proof number 0 --- there is nothing left to
proof. Nodes with value \value{false} have proof number $ \infty $ there
exist no smaller set of nodes which proof it. Leaf nodes with value \value
{unknown} have proof number 1 because there is enough to prove the node itself.
Internal \node{and} node has proof number which is equal to sum of proof numbers
of its son, because for proving the \node{and} node we must prove all its sons
by proving their proof sets. Internal \node{or} node has poof number equals to
minimum of proof numbers of it sons because we need to prove one of its son.

Determining of disproof number works similar way. As you can see an picture
\obr{3)}. 

\newtheorem*{mostProvingNode}{Definition}	
\begin{mostProvingNode}
	For any game tree $T$ a~{\sl most-proving node} of $T$ is a~node, which by 
	obtaining the value \value{true} reduces T's proof number by 1, while by obtaining the
	value \value{false}  reduces T's disproof number by 1.
\end{mostProvingNode}

Most-proving node is node which helps solve root in both cases (proving or disproving it)
so we will use 
it as the best node in best-first search. We will show that such node exist in each
partial tree.

\begin{proof}

	We will prove it by induction on depth of partial game tree.
	It is enough there are such minimal proof and minimal disproof sets that they have 
	common node. Proving the node reduces the proof set and disproving it reduces
	disproof set.

	\begin{itemize} 
		\item{Basic} 
			If root is a leaf node both proving and disproving set 
			are the root.
		\item{Induction step}
			Let root be an \node{and} node. Proof set of the root is proof set
			of it's son which has minimal cardinality. Disproof set of the root
			is union of disproof sets of it's sons, so it contains
			disproof set of the son which has minimal proof set.
			Imagine a game (sub)tree where this son is root. By induction his
			proof and disproof sets have common node, which is also 
			in intersection of proof and disproof sets of original root and
			is the most proving node.
			Proof for \node{or} node proceeds analogously.
	\end{itemize}
\end{proof}

Now we can describe whole pn-search. We store partial game tree and we will
store (dis)proof numbers of each node. We start with one node --- root. We will
repeat three steps until root is solved. First step is
\emph{SelectMostProving(node)}. This function works similar as the proof that
most proving node exists. \obr{4)obrazek} We start in root and continue by
choosing the sons which have minimal proof number when we are in \node{and}
node disprove number in \node{or} node until we are in a leaf node. When there
are more nodes with minimal number we choose leftmost. Second step is
\emph{DevelopNode(mostProvingNode)}. There we create sons and evaluate them.
Third step is \emph{UpdateAncestors(mostProvingNode)}. Proof and disproof
numbers in partial game tree have changed so we must update their stored
values. We start updating mostProvingNode and continue by updating its
ancestors up to the root. At the end of the iteration the (dis)proof numbers
are again correct, because only numbers above mostProvingNode could have
changed.

\TODO{opsat pseudokod z allis}

\subsection{Choice of algorithm}

\TODO{prepsat}
ze tetusim jestli je vyhr/prohr u nejakych nahodnych nodu
We choose pn-search to our problem. It seems to be quicker then alpha-beta. \TODO{citace} Alpha-beta
could be part of pn-search when we use rating from it as a heuristic (\TODO{11}, \ref{ord}).
And finally there are many enhancements and heuristics which seem to help reduce the partial game tree size  
then the choice the algorithm.

\section{Enhancements of PN-search}

\TODO{obecny kec ze to
	dela prakticky jine algoritmy 
	a ze muze byt problem s kombinovanim,
	ze vetsina byla v allise at nemusim citovat furt,
ze pd-numbers MPN nejsou podle definice
ze to aplikuju hlavne na ten zakladny
ze tady hlavne predstavuju jak funguji neresim jak moc pomuzou
}

\subsection{Path from root}

Proof number search is best-first search. Its disadvantage is that searching for most-proving node takes
a lot of time. 

\newtheorem*{currentPath}{Definition}	
\begin{currentPath}
	A~{\sl current path} is a~path from root to the  most proving node. \TODO{ktera cesta}
\end{currentPath}

In pn-search each iteration starts at the root and descends down. After developing
most proving node we update proof and disproof numbers starting in most-proving and 
return the way back into root. So we traverse the current path twice. 

There are two enhancements which reduce the number of nodes traversed to select the
most proving node. 

\subsubsection{Last changed node} \label{last}

We make two basic observations.
\begin{itemize}
\item When there is no change during node update, traversal can be stop even though we
aren't at root.
\item Current paths of two consecutive iterations have the same beginning. They can differ
only in part where proof or disproof number was updated.
\end{itemize}

\newtheorem*{currentNode}{Definition}	
\begin{currentNode}
A~{\sl current node} is a~node from current path. It is the lowest unchanged node or
root if everything was changed.
\end{currentNode}

\TODO{zkontrolovat jestli sedi nazvy fci}

Enhanced pn-search works like ordinary pn-search. It has one variable
called $currentNode$, at the beginning value of $currentNode$ is root. Each
$selectMostProvingNode()$ starts in $currentNode$. And each $updateAncestors()$
ends at the node where proof and disproof numbers haven't been changed and sets 
$currentNode$ to the last changed node. You can see as it works there. \obr{(5))}

\subsubsection{DF-PN search} \label{dfpn}

We can save even more traversing. In two consecutive iterations current paths
have common beginning. \obr{6.1)} We reduced the part when proof and disproof
numbers stay same. Now it's time to reduce part when they are changing and work
as in \obr{6.2b)}. We need to know the highest node which will not be in new
current path. We use thresholds for it --- we will store two new numbers for
each node on the current path. They will be set so that when (dis)proof number
will be smaller than its threshold we stay in the same subtree. If its equal or
bigger we must either update also its parent or the son with minimal (dis)proof
number will by different. 

Let $p$ and $d$ be proof and disproof number of node $n$. Let $pt$ and $dt$ be
their thresholds. When $n$ is \node{and} node we assume that its sons $1\ldots k$
are ordered so $p_1 < p_2 < \ldots < p_k$. Rules for setting new thresholds
were deduced in \TODO{epsilon trick str4}:\obr{6.3)} 
\begin{eqnarray*} 
	pt_1 = min(pt, p_2+1), dt_1 = dt - d + d_1.
\end{eqnarray*}

When $n$ is \node{or} node we assume that $d_1 < d_2 < \ldots < d_k$. Rules
are symmetrical:  
\begin{eqnarray*} 
	pt_1 = pt-p+p_1, dt_1 = min(dt,d_2+1).
\end{eqnarray*}

When we delay updateAncestors to the time when $p > pt$ or $d > dt$ we can make
another changes. 

Main point is that we don't need to store whole partial game three. It is
enough to store nodes from current path. Algorithm could be implemented
recursively (but we didn't done it) and for this reason it is called depth-first.
It isn't typical depth-first search because one node can be visited more
then once \TODO{je tu potreba priklid nebo to je videt?}. 

DP-PN search is typically implemented with cache used for storing as many
nodes from partial game tree as possible. 

\subsection{Transpositions in DAG} \label{DAG}
 
As we see in picture \obr{8) 3 obr,neco jen ... misto hran, nespojene, spojene
jen poradi, spojene i symetrie popisek..} of chess tree we are exploring one
game state many times. Basic enhancement is to join that nodes into one.

The problem is that we have game DAG \footnote{There aren't cycles because in a
positional game all edges leads into position with more positions occupied.}
instead of game tree and proof and disproof numbers cannot by computed like
above because one node can increase proof number of another node more than by
one \obr{7)obr Proof and disproof numbers a) by definition --- disproof set for
root are three undeveloped nodes b) computed as
above}. However $proofNumbers$ and $disproofNumbers$ computed by
PN-search can be useful even though they can be higher then proof and disproof
number as they was defined. 

We can use PN-search with a few modifications. When we generate children we
check whether they already exist. We can use hash table for this. Second
modification is that we need to update all parents. This could cause problems
if used together with other enhancements as last changed node \ref{last} but if
we use ordinary PN-search it works. Allis proof it in \cite{allis} (page
39-40).

One game state in turn $t$ can be visited in the worst case $t!$ times because
positions can be occupied in any order. So joining them is a good idea and
there exist many enhancements of this enhancement.

In this thesis we will try to use advantage of symmetry as much as possible.
So we want in addition join game positions which are isomorph. See \obr{9) }
how it works on tic-tac-toe game tree. However in clique game we cannot joint
all isomorph game position into one because finding \TODO{jak se to pise?}
represent is hard problem. So we will use almost representative graph instead.
See section \ref{norm}.

\textbf{Note}: We will use "game tree" even if we talk about game DAG. It can be little confusing 
but in many cases there isn't any reason to distinguish if we talk about tree or DAG. 
We will use "leaf node" when we are talking about undeveloped node. 

%\subsection{Heuristic 1 1}
%%
%\TOD{napsat co je me?}
%
%By definition the proof and disproof numbers of a leaf node with value
%\value{unknown} are 1 and 1. In general and/or tree we don't now anything more
%and we must use this assignment, when we know more we can set these numbers
%differently and break definition.
%
%The way the algorithm proceeds can seems similar to breath-first search when we run it
%on some tree where up to some depth initially all nodes have value \value{unknown}
%and each node on same level has same number of sons. \ODO{obrazek vyvoj}
%
%For this reason it is useful to initially set proof and disproof numbers
%differently. First we can use the numbers which the node will probably have if
%we develop it. For example if the \node{and} node has $n$ sons we give it proof
%number $n$ and disproof number $1$.
%
%We will call this enhancement $n 1$. It causes the algorithm will develop left most
%nodes until it finds some node with value \value{true} or \value{false}. \ODO{obrazek}
%
%\ODO{!!!!!!!!!!!!!!!!!!!!je tohle dobre? ? myslim ze je}
%There is other way how to assign the numbers. \ODO{citace} When PN-search works on partial
%tree where are only few or no value \value{true} and \value{false} nodes, it prefers node with 
%fewer sons. We can initially set $n$ using some heuristic. When it is \node{and} node with
%$n$ sons and position is good for first player we give it numbers $m 1$ ($m$ is slightly less
%then $n$). When it is \node{or} node with $n$ sons and position is bad for first player we
%give it numbers $1 m$ ($m$ is slightly more then $n$).

\subsection{Weak PN-search}

When we use transposition into DAG one node can by counted in (dis)proof number many times.
Hence we show an enhancement which tries to minimize this disadvantage. 

\TODO{tady mam v kodu nejakou vyjimku to je pro 0 nebo pro 1?}
\TODO{zkontrolovat ze je tam ta \^2 jen jednou}
\TODO{da se to cele nejak jednoduse vysvetlit? \^2.. kdyz vyvracim zalezi na branching factor kdyz dokazuju prehodit}
We will count proof and disproof number in a different way:
\begin{enumerate} 
	\item Let $n$ be a leaf node. It depends on value of node.
		\begin{itemize}
			\item \value{true} $p(n)=0$, $d(n)=\infty$
			\item \value{false} $p(n)=\infty$, $d(n)=0$
			\item \value{unknown} $p(n)=1$, $d(n)=1$
		\end{itemize}
	\item Let $n$ be an internal \node{or} node with $k$ sons \newline
		$p(n) = min_{1 \le i \le k}(n_i)$ \newline
		$d(n) = max_{1 \le i \le k}(n_i) + (k-1)^2$ 
	\item Let $n$ be an internal \node{and} node with $k$ sons\newline
		$p(n) = max_{1 \le i \le k}(n_i) + (k-1)$ \newline 
		$d(n) = min_{1 \le i \le k}(n_i)$
\end{enumerate}
		
\subsection{Deleting solved subtrees}

To save memory we can delete node after it is solved. However this doesn't 
help --- see \TODO{4.kapitola}

\subsection{ Prohrano pokud neni vyhravajici linie}
\TODO{dopsat az po prvni kapitole}

\subsection{PN-set}
\TODO{dopsat az napragramuju}

\subsection{1+\TODO{epsilon} trick}
\TODO{dopsat az napragramuju}


\TODO{tady mozna PN2, DB}


\chapter{ Application algorithms on clique game }

\TODO{tady nejaky obecny kec}

\section{ Solution \TODO{obteceni} }

\subsection{ Start from root}

\subsection{ Start at the highest changed level}

\subsection{ Remember whole path from root }

\section{ Deleting mutiedges }

\section{ Implementation of update ancestors function}

\section{ Clique game - graph representation }

\subsection{ Triangle in line }

\subsection{ \TODO{scvrkavani} }

\subsection{ Adjacency matrix in line}

\subsection{ Trick: Find common neighbours }

\subsection{ Find K4 of one color }

\subsection{ Find \TODO{vyhravajici linie} }

\subsection{ Find threat }

\section{ Clique game - \TODO{norm} }

\section{ Heuristic sequence of child }

\section{ Our implementation }

\TODO{jazyk, rocnikac, blbosti}

\subsubsection{ Used enhancements }

\subsubsection{ Statistic }
\TODO{kdo co travi kolic casu v nejrycchlejsi verzi}

\subsubsection{ Licence }






\chapter{Software results}

In this chapter we will talk directly about our softwere. We give some
statistic which ilustrate which ehantments helps with solving clique
game. 

\section{ Clique game }
We solved clique game for $K=4, N=6$ and $ N=7$. In this section we
describe partial game trees which was created during solving that
games. We use \sec{weak pn-search} \ref{weak} and \sec{transposition
into DAG} \ref{DAG} with norm function \ref{norm2} and \TODO{\sec{
nofreeK4}} \ref{nofreeK4}.
 
%N 7 M 22
%CACHE_SIZE 16777216, CACHE_PATIENCE 100, MAXNODES 6000000
%DEBUG WEAK STATS NORM2 TURNDDELETECHILDRENST 0 NOFREEK4 UPDATEANCESTORS2 

\begin{table} 
\begin{tabular}{l|c|c|c}
level & \sec{all} & \sec{created} & \sec{solved} \\
\hline
$\sum$ & 10165779& 1241 (0,12)& 110\\
0& 1& 0 (0,0)& 1\\
1& 16& 1 (0,0)& 1\\
2& 240& 2 (0,0)& 1\\
3& 1680& 7 (0,0)& 6\\
4& 10920& 26 (0,0)& 5\\
5& 43680& 73 (0,0)& 28\\
6& 160160& 190 (0,0)& 14\\
7& 400400& 306 (0,0)& 31\\
8& 900900& 273 (0,0)& 23\\
9& 1441440& 363 (0,12)& 0\\
10& 2018016& 0 (0,0)& 0\\
11& 2018016& 0 (0,0)& 0\\
12& 1681680& 0 (0,0)& 0\\
13& 960960& 0 (0,0)& 0\\
14& 411840& 0 (0,0)& 0\\
15& 102960& 0 (0,0)& 0\\
16& 12870& 0 (0,0)& 0\\
\end{tabular}
\caption{Property of game tree of the clique game ($N=6,K=4$).}
\label{stats6}
\end{table}

\begin{table}
\begin{tabular}{l|c|c|c}
level & \sec{all} & \sec{created} & \sec{solved} \\
\hline $\sum$ & 6 377 181 825& 65 979 (92, 605)& 2 356\\
0& 1& 0 (0, 0)& 1\\
1& 22& 1 (0, 0)& 1\\
2& 462& 2 (0, 0)& 2\\
3& 4 620& 7 (0, 0)& 7\\
4& 43 890& 29 (0, 0)& 7\\
5& 263 340& 107 (0, 0)& 44\\
6& 1 492 260& 396 (0, 0)& 26\\
7& 5 969 040& 1 247 (0, 0)& 242\\
8& 22 383 900& 3 812 (0, 0)& 76\\
9& 62 674 920& 9 108 (0, 49)& 752\\
10& 162 954 792& 16 938 (0, 96)& 226\\
11& 325 909 584& 17 303 (9, 113)& 452\\
12& 597 500 904& 8 252 (0, 119)& 293\\
13& 853 572 720& 7 280 (43, 203)& 159\\
14& 1 097 450 640& 825 (0, 0)& 22\\
15& 1 097 450 640& 672 (40, 25)& 46\\
16& 960 269 310& 0& 0\\
17& 640 179 540& 0& 0\\
18& 355 655 300& 0& 0\\
19& 142 262 120& 0& 0\\
20& 42 678 636& 0& 0\\
21& 7 759 752& 0& 0\\
22& 705 432& 0& 0\\
\end{tabular}
\caption{Property of game tree of the clique game ($N=7,K=4$).}
\label{stats7}
\end{table}

In the first case ($N=6$) algorithm does 337 iteration to solve the game. It the second
case algorithm does 16489 iteration and solve it in tree second.

Tables \ref{stats6} and \ref{stats7} show information about levels in trees.

The column \sec{all} illustrate how many game state could exist on some level $L$. This
number was computed as a number of edge coloring graph with $N$ nodes with $L$
\fce{div} 2 red edges and $(L-1)$ \fce{div} 2 blue edges. So numbers from the first
column are not exactly right because they count game states which have for example
two $K_4$ with same color which cannot happen in real game.

In the column \sec{created}  are number of nodes which was really created on each level. 
In the bracket behind there how many of them was first evaluated as a \value{false} respectively
how many of them are thread. There was no node which was first evaluated as \value{true},
this is a god mark, because nodes with value \value{true} never helps to disprove anything and
algorithm se jim vyhne.

In the column \sec{solved} there is how many of nodes was solved on its level.
All of them was solved as a \value{false}.

\section{ \TODO{Meření zlepšení} }

\TODO{obecny kec o mereni
nezapomenout originalni cisla
}

\subsection{df}
\TODO{ze df-pn moc nepomohlo ale usetrilo pamet
df-proc mi moc nepomuze ze nezalezi na ceste ale na norm 
nefunguje nofreeK4
}

\subsection{Weak pn-search}
\TODO{
zkusit jine kombinace \^2
}

\subsection{Deleting solved subtrees} 
\TODO{nepomohlo, ted uz neni}

\subsection{\TODO{no free K4}}

\subsection{Ordering of sons} 
\TODO{nepomohlo, lepsi heuristiky?}

\subsection{Deleting multiple edges}

\subsection{Cache}

\subsection{\fce{selectMostProvingNode()} and \fce{updateAncestors()} }
\TODO{jak dlouho za jakych okolnosti trvna select update}

\subsection{\TODO{mereni obteceni}}

\section{ More about our implementation }

Our software was created in order to solve clique game for $K=4$ and as higher
$N$ as is possible. We knew that there is no chance to solve it for all $N$ up
to 17 because only check if winning strategy is right needs to check so many
states that it could be done. 
 
We chose language C. Our code is on github \TODO{jak se sazi adresa?} and
it is under \TODO{licence - i udelat}.

During crating this project we need to test if some heuristic helps or not.
First we try to implemented something and after we decide if let it there or if
deleted it. Later we decided to write enhancement into ifdefs. The result is
that we can try combination of more enhancement and we can compare them.
There are thousands of combination. Not all of them works perfectly. List
of combination which was tested is in documentation of the project. 

\subsection{ \TODO{nejrychlejsi verze} }

\TODO{kde je pomala}

\section{ Futher work }

\TODO{ i na teorii 
tabulky- zbytecne moc vytvorene
}

%\include{example}

\chapter*{Conclusion}

In this thesis we introduced algorithms for solving strong positional games,
mainly PN-search and its enhancements.

We focused on games with high symmetry. The high symmetry allows, by joining
several isomorph game state into one, to narrow the partial game tree. When
partial game tree is narrow, we can search in it deeper. The other feature which
keeps the tree narrow is that we do not deal with high branching factor. For
example, we can compare our game with the game Tzaar, which has high branching
factor and low symmetry. Despite of clever heuristics and optimizations the
best bot playing the game searches the game tree only into depth
4~\cite{paulie}. For comparison, one variant of our solver for (7,4)-clique
game solved the game tree to depth 16.

There are other features distinguishing games with high symmetry. It is much
more important to join the nodes. We must treat carefully the lifetime of
the nodes, that is, we do not want to delete nodes from cache especially when their
solution was expensive because as we may find another ancestor of these nodes.
Finally in games with high symmetry much more often happens situation described
in Figure~\ref{deset}.

\subsubsection{Further work}

There are many enhancements which we did not implemented and even do
not mention. For example 1+$\epsilon$ trick.

It could also be useful to implement enhancement ``dveloping a~node without creating sons''
in Section~\ref{lastenh} and study if it can significantly reduce the partial game
tree. Also to might it may be interesting to try it on other games.

Of course, there is space for more new enhancements and heuristics. 

\addcontentsline{toc}{chapter}{Conclusion}



%%% Seznam použité literatury
%%% Seznam použité literatury je zpracován podle platných standardů. Povinnou citační
%%% normou pro bakalářskou práci je ISO 690. Jména časopisů lze uvádět zkráceně, ale jen
%%% v kodifikované podobě. Všechny použité zdroje a prameny musí být řádně citovány.

\def\bibname{Bibliography}
\begin{thebibliography}{99}
\addcontentsline{toc}{chapter}{\bibname}

\bibitem{lamport94}
  {\sc Lamport,} Leslie.
  \emph{\LaTeX: A Document Preparation System}.
  2. vydání.
  Massachusetts: Addison Wesley, 1994.
  ISBN 0-201-52983-1.
%
%\bibitem{allis}
%  {\sc Allis,} L.V. and others.
%  \emph{Searching for solutions in games and artificial intelligence}.
%  \TODO{je treba vydání?!!}
%  Wageningen: Ponsen \& looijen, 1994.
%  ISBN 90-9007488-0.

\end{thebibliography}


%%% Tabulky v bakalářské práci, existují-li.
%\chapwithtoc{List of Tables}

%%% Použité zkratky v bakalářské práci, existují-li, včetně jejich vysvětlení.
%\chapwithtoc{List of Abbreviations}

%%% Přílohy k bakalářské práci, existují-li (různé dodatky jako výpisy programů,
%%% diagramy apod.). Každá příloha musí být alespoň jednou odkazována z vlastního
%%% textu práce. Přílohy se číslují.
%\chapwithtoc{Attachments}

\openright
\end{document}
