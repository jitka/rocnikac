\chapter{Games}

In this chapter we introduce  combinatorial games and define several types of
them. We will describe one such game in more details. 

\section{Definitions}

We explain the basic combinatorial game definition informally on the example of
chess. There are two \emph{players} which alternate in their moves. There is
a~board with 64 squares, some pieces placed on some squares and some player is
on turn. We call this the \emph{game state}. The player \emph{turns} by moving
some pieces to some other game state. 
\TODO{pravopis}
He decides how turn from a game state. All decision from each game state where
is on turn is \emph{strategy}.
When one of the players moves to a~\emph{winning
state} called mate, he wins. Both players want to win. There are also \emph{draw
states}, this is for example when there is no legal move. 

\subsection{Positional games}

In \emph{positional games}, the game state is set of states of
\emph{positions} which are either \emph{free} or \emph{claimed} by a~player.
Tic-tac-toe is a~typical positional game. There are nine square positions. In
each turn one of the two players claims one free position for example by drawing a
cross. The rules of the game define a~set of positions called the \emph{winning
set}. In tic-tac-toe, every straight line with three squares is a winning set.
Winning set is \emph{claimed} by a~player when the player claimed all positions
from the set. Winning set is \emph{free} for a player when no position in it is
claimed by his opponent. The situation when all except one position of
a~winning set is claimed by a~player and the last position is free is
a~\emph{threat}.

There are two types of positional games. In the last paragraph we described a
\emph{strong game}. In a~strong game, both players want to claim a~winning set
and who claims first wins. In the other variant, called \emph{maker-breaker
game} only the first player wants to claim a~winning set. The second player wins in the
case that every position is claimed and the first player has not claimed a~winning
set. There is no draw in the maker-breaker variant.

There are three possible endings of a~strong positional game: the first player
wins, the second player wins, there is a~draw. As we will show in Theorem
\ref{stealingStrategy}, if both players play their best possible strategies,
the second player cannot win. The main question of this thesis is to find
algorithm which \emph{solves} the game, i.e., determine if there is a~winning
strategy for first player or a~draw strategy for the second player in the
finite strong positional game.

\newtheorem{theorem}{Theorem} 
\begin{theorem} \label{stealingStrategy} 

In a strong positional game the second player cannot has a winning strategy.

\end{theorem}

\begin{proof}

We prove it by contradiction. Suppose that the second player has a winning strategy
$s$. We show a winning strategy $s'$ for the first player. 

The first player claims a random position $p$ in his first turn. In later
turns, the first player imagines that the position $p$ is free and he plays
according to the strategy $s$. When the strategy $s$ says to claim $p$, he claims
another random free position $p'$ and from this moment, he will use $p'$ instead
of $p$.

The first player also has a~winning strategy, but both players cannot win.
Contradiction. 

\end{proof}

There are several algorithms for solving general combinatorial games. We apply
them to a~strong positional game. Description of these algorithms is simpler.
The algorithms are usually designed for games with \emph{high branching factor}
(that is the average number of possible moves in each game state is high --
a typical example of such a game is Go) and \emph{low
symmetry} (that is only few or no distinct game states are symmetrical). 


\section{Clique game}

In this section we define the clique game which we are going to use as the
main example in this thesis.

\subsection{Rules}

The clique game is a strong positional game. 

Two players play on a clique graph with
$N$ vertices. In each turn, one player colors one edge by his color. The first
player has red color and the second player blue color. The player who colored
a~clique subgraph on $K$ vertices wins. 

We call it ($N$,$K$)-clique game.
See example of (5,3)-clique game game for in Figure \obr{-1)}.

Solution of ($N$,4)-clique game for $N < 18$ are open problem mentioned
by Beck~\cite{beck}.

\subsection{Solution for some N, K}

For some $N$ and $K$, the game can be solved by purely theoretical argument
without computer backtracking.

We will use Ramsey's theorem. There are many variants of Ramsey theorem, the weakest
one is enough for us. For proof see, e.g. Diestel~\cite{ramsey}. 

\begin{theorem} 
	
For any positive integer $k$, there exists a~positive integer $n$ such that for
any clique on at least $n$ vertices whose edges are colored red or blue, there
exists either a~clique on $k$ vertices with has all edges red or has all edges
blue.

\end{theorem}

\begin{theorem} \label{solution} 
	
For fixed $K$, there exists $N_0$ such that in each ($N$,$K$)-clique game where
$N > N_0$ is a winning strategy for the first player.  

\end{theorem}

\begin{proof} 
	
We chose $N_0$ as a~$n$ in Ramsey theorem. We already proved that either the
first player has a winning strategy or the second player has draw strategy. It
suffices to prove that there is no draw in ($N$,$K$)-clique game. Draw game
state is a game state where every edge is claimed -- it has a~red or blue color
and there is no clique on $K$ vertices with all edges of the same color.
However, this is not possible since according to the Ramsey's theorem, there is
always such a clique. So there is no draw in ($N$,$K$)-clique game.

\end{proof}

\subsubsection{The case $K=2$}

For every $N \ge 2$, the first player has a winning strategy. He just claims one edge
and wins.

\subsubsection{The case $K=3$}

The number $n$ from Ramsey's theorem 6 for $k=3$. So for $N \ge
6$ the first player has a winning strategy, as we proved in
Theorem~\ref{solution}. 

When $N=5$, the first player also has a~winning strategy. He plays similarly as in
Figure \obr{-1}. First he marks edge $(u,v)$. When we count symmetry the second player has
two possible moves to $(v,w)$ and to $(w,x)$. The first player turns to $(u,y)$ and
creates a~threat. The second player must play $(y,v)$. The first player turns to
$(u,x)$ which creates two threats $(x,v)$ and $(x,y)$ and he is going to win. 

When $N \le 4$ the second player has a~draw strategy which can be found by simple
backtracking over all position manually. 

\subsubsection{The case $K=4$}

When $N \le 5$ the second player has a~draw strategy. There are not enough edges to
claim the whole $K_4$.

When $N \ge 18$, the first player has a~winning strategy as we proved in Theorem
\ref{solution}.

An interesting question is what is the lowest $N$ for which the first player has
a~winning strategy. Our solver proved that for $N=6$ and $N=7$ the second
player has a~draw strategy.

