\chapter*{Introduction}
\addcontentsline{toc}{chapter}{Introduction}

This thesis introduces an algorithm for solving strong positional games
with high symmetry. We describe the most important algorithm for solving 
combinatorial games -- PN-search and its various enhancements which change
it so much that they can be considered new algorithms. 

%First, we present several known enhancements. Second, some enhancements
%which we designed. Then we test them on the example of the clique game.
%
%Our solver proved that in (6,4) and (7,4)-clique game the second player has a draw
%strategy. This was expected but not proved yet.

\medskip

PN-search was introduced by Allis~\cite{Allis} in 1994. It has become well know algorithm
and basis for a lot of further work. It was successfully applied for a wide class
of games and many enhancements was invited. We introduce some of them and how they can be useful
in case of a game with high symmetry. We consider which  enhancements
can be used together and what problems can arise in combinations.
We also introduce some enhancements which we discovered independently.

\medskip

We do not determine neither asymptotic time complexity nor asymptotic space complexity of the algorithms in this thesis.
It is not clear what time/space complexity means in games solving -- 
if the input of the  algorithm are rules of the game most of games can be solved in constant time.

On the other hand when the input of the algorithm is a game state
of a positional game with $n$ positions solving the game is at least hard problem.
($n = {N \choose 2}$ in ($N$,$K$)-clique game). The number of game states in
such a game is \TODO{bigO} $O(2^{N \choose 2})$. It is not necessary to visit them all,
however, the number of the game states we must visit even for proving that a strategy
is winning is also $O(2^{N \choose 2})$. From this view all improvements in this
thesis are just heuristic.

The size of the partial game tree which we must create for solving the game is 
much more important then asymptotic complexity.

\subsubsection{Outline}

In the first chapter we informally introduce some combinatorial
games terminology. We suppose the reader has general knowledge about graph theory, 
generally see Deisel \ref{TODO}. In the first chapter we also define clique game and give
solutions of some cases of a clique game.

Known algorithms and their enhancements are introduced in the second chapter. Usually
algorithms are designed for general combinatorial game. We apply them to
strong positional games, which simplifies the descriptions. We start with a meta
algorithm of game solving. Some specific algorithms are described after. We choose PN-search to
implement and to be described in detail.

Theoretical results which we found independently are described in the third chapter.
In subsections \ref{} to \ref{} TODO are enhancements of PN-search. Then we introduce some rules for
deleting from cache which are designed for strong positional games. We consider
a problem which can occur when we combine some enhancement and we
propose solution for the problem. In the end of the chapter we focus on
a graph of a clique game \TODO{toto do dreti kap}. We describe several ways of storing
the game state and also how to normalize them.

The last chapter introduces our software and presents experimental results.
First, we describe the partial game tree created by our solver during solving
(6,4) and (7,4)-clique games. Then we compare some mentioned enhancements. We
study if they help to solve clique game. In the end of the chapter we describe in
more detail the software implementation.
